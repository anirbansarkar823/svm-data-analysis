\documentclass[oneside,twocolumn]{article}
\usepackage{mathtools}

%% ADD PACKAGES %%

\title{Attributes tagging for Ration Shop/ Fair Price Shop}
\author{Kipa Tachak}
\date{Sunday  9 June 2019}

\setcounter{secnumdepth}{3}
\newcommand{\tsub}[2]{\text{#1}_{\text{#2}}}
\newcommand{\tsubb}[2]{#1_{\text{#2}}}
\newcommand{\dsub}[2]{\dfrac{\text{#1}}{\text{#2}}}
\newcommand{\multsel}[1]
{
	\[
		\tsub{Score}{#1} = \dsub{Sum of selected}{Sum of ranks}
	\]
}
\newcommand{\singsel}[1]
{
	\[
		\tsub{Score}{#1} = \dsub{Rank of selected}{Max. rank}.
	\]
}
\newenvironment{ttable}
{
\begin{center}
\begin{tabular}{c|c}
\hline
}
{
\\ \hline
\end{tabular}
\end{center}
}

\begin{document}
\maketitle

%% BODY %%
\section{About the document}
This file documents the process of asset tagging for the asset 'Ration Shop/ Fair Price Shop' in the
vertical 'Food and Civil Supplies'.

\section{Attributes and questions}
The list of sections, variables, and questions are:
    \begin{itemize}
    \item Observation
    \item General
    \item Asset Specific
    \begin{itemize}
\item Tech Scope
\begin{enumerate}
\item Digital Technology: Do you have a POS device in your shop?
\end{enumerate}

\item Availability
\begin{enumerate}
\item Maintenance: What is the frequency of getting supply for your shop?
\item Services: What are the foodgrains/items available in the fair price shop?
\end{enumerate}

\item Technology Used
\begin{enumerate}
\item Transparency: How do you weigh any goods?
\end{enumerate}

\item Outreach
\begin{enumerate}
\item Market Reach: How many people (villagers) are enrolled as beneficiaries in the fair price shop?
\end{enumerate}

\item Potential
\begin{enumerate}
\item Market Reach: How many villages do your fair price shop cover?
\end{enumerate}

\end{itemize}

    \end{itemize}
\section{Overall scoring}
The overall scoring of the asset is given as,
\begin{align*}
	\tsub{Score}{asset} &= \tsubb{W}{observation} \times \tsub{Score}{observation} \\
	&+ \tsubb{W}{general} \times \tsub{Score}{general} \\
	&+ \tsubb{W}{specific} \times \tsub{Score}{specific}.
\end{align*}

The score of a section or a variable is given as,
\[
	\tsub{Score}{var} = \sum_{l \in P} \tsubb{W}{l} \times \tsub{Score}{l}.
\]
where,
\[
	P = \text{Set of sub-parameter labels.}
\]
\section{Attribute scoring and the
process}
\subsection{Asset Specific}
\subsubsection{Tech Scope}

\paragraph{Digital Technology: Do you have a POS device in your shop?}

The score is given as,
\begin{align*}
\tsub{Score}{Digital Technology} &= 0, \text{If no} \\
        &= 1, \text{If yes}.
\end{align*}
\subsubsection{Availability}

\paragraph{Maintenance: What is the frequency of getting supply for your shop?}

The options given are,
\begin{ttable}
Options & Rank \\ \hline
Weekly & 5 \\
Monthly & 4 \\
Quarterly & 3 \\
Half Yearly & 2 \\
Yearly & 1 \\
\hline
\end{ttable}
The score is given as,
\singsel{Maintenance}
\paragraph{Services: What are the foodgrains/items available in the fair price shop?}

The options given are,
\begin{ttable}
Options & Rank \\ \hline
Rice & 1 \\
Wheat & 2 \\
Flour & 3 \\
Sugar & 4 \\
Kerosene & 5 \\
\hline
\end{ttable}
The score is given as,
\multsel{Services}
\subsubsection{Technology Used}

\paragraph{Transparency: How do you weigh any goods?}

The options given are,
\begin{ttable}
Options & Rank \\ \hline
Manual & 1 \\
Electronic & 2 \\
\hline
\end{ttable}
The score is given as,
\singsel{Transparency}
\subsubsection{Outreach}

\paragraph{Market Reach: How many people (villagers) are enrolled as beneficiaries in the fair price shop?}

The score is given as,
\begin{align*}
\tsub{Score}{Market Reach} &= 1, \tsubb{N}{villagers} \ge \text{Threshold} \\
        &=
\Big(\dfrac{\tsubb{N}{villagers}}{\text{Threshold}}\Big)^{2}
\text{Otherwise}.
\end{align*}
\subsubsection{Potential}

\paragraph{Market Reach: How many villages do your fair price shop cover?}

The score is given as,
\begin{align*}
\tsub{Score}{Market Reach} &= 1, \tsubb{N}{villages} \ge \text{Threshold} \\
        &=
\Big(\dfrac{\tsubb{N}{villages}}{\text{Threshold}}\Big)^{2}
\text{Otherwise}.
\end{align*}

\end{document}
