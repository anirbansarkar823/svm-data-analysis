\documentclass[oneside,twocolumn]{article}
\usepackage{mathtools}

%% ADD PACKAGES %%

\title{Attributes tagging for Emporium}
\author{Kipa Tachak}
\date{Thursday  6 June 2019}

\setcounter{secnumdepth}{3}
\newcommand{\tsub}[2]{\text{#1}_{\text{#2}}}
\newcommand{\tsubb}[2]{#1_{\text{#2}}}
\newcommand{\dsub}[2]{\dfrac{\text{#1}}{\text{#2}}}
\newcommand{\multsel}[1]
{
	\[
		\tsub{Score}{#1} = \dsub{Sum of selected}{Sum of ranks}
	\]
}
\newcommand{\singsel}[1]
{
	\[
		\tsub{Score}{#1} = \dsub{Rank of selected}{Max. rank}.
	\]
}

\newenvironment{ttable}
{
\begin{center}
\begin{tabular}{c|c}
\hline
}
{
\\ \hline
\end{tabular}
\end{center}
}

\begin{document}
\maketitle

%% BODY %%
\section{About the document}
This file documents the process of asset tagging for the asset 'Emporium' in 'Hand loom and Handicrafts'.

\section{Attributes and questions}
The sections Observation and General are similar to other assets. Hence we will be only discussing the specific section in this document.

The list of variables and questions are given as:
\begin{itemize}
	\item Initial Observation.
	\item General
	\item Specific
		\begin{itemize}
			\item Basic Info
				\begin{enumerate}
					\item Description: Which of the following products are sold in the emporium?
				\end{enumerate}
			\item Business specific
				\begin{enumerate}
					\item Revenue: What is the month revenue generated?
					\item Operational Cost: How must is the operations cost of the emporium?
				\end{enumerate}
			\item Customers
				\begin{enumerate}
					\item No. of customers: How many customers do you get in a month?
					\item Customer type: Who among the above selected are you major customers?
				\end{enumerate}
			\item Operations
				\begin{enumerate}
					\item Suppliers: Whom among the following do you procure the finished products from?
					\item Procurement: Do you procure products from artisans in bulk?
					\item Order placement: How can orders be placed for the products in the emporium?
					\item Payment mode: Which mode of payment do you accept?
					\item Accessibility: How is the products from the artisans brought to the emporium?
					\item Accessibility: How many outlets do you have in town and cities?
				\end{enumerate}
			\item Market potential
				\begin{enumerate}
					\item Product reach: Are your products being displayed at exhibitions and expos?
					\item Market potential: Do you get bulk orders for you products?
					\item Market potential: Are the products sold in your emporium branded?
				\end{enumerate}
			\item Customer services
				\begin{enumerate}
					\item Services: Do you provide delivery services for your products?
					\item Catalog: Do you have a cataloger of your products?
				\end{enumerate}
		\end{itemize}
\end{itemize}

\section{Overall scoring}
The overall scoring of the asset is given as,
\begin{align*}
	\tsub{Score}{asset} &= \tsubb{W}{observation} \times \tsub{Score}{observation} \\
	&+ \tsubb{W}{general} \times \tsub{Score}{general} \\
	&+ \tsubb{W}{specific} \times \tsub{Score}{specific}.
\end{align*}

\section{Attribute scoring and the process}
\subsection{Specific}
\subsubsection{Basic Info}
\paragraph{Description: Which of the following products are sold in the emporium?}
The options given are,
\begin{ttable}
	Items sold & Rank \\ \hline
	Others & 1 \\
	Items of daily use & 2 \\
	Clothes & 3 \\
	Decors and home furnishing & 4 \\ \hline
	Sum of ranks & 10
\end{ttable}
The score is calculated as,
\multsel{basic info}
\subsubsection{Business specific}
\paragraph{Revenue: What is the month revenue generated?}
The score for revenue generated is given as,
\begin{align*}
	\tsub{Score}{revenue} &= 1, \tsubb{R}{rupees} \ge \text{Threshold} \\
	&= \Big( \dfrac{\tsubb{R}{rupees}}{\text{Threshold}} \Big)^{2}, \text{Otherwise}.
\end{align*}
where,
\[
	\tsubb{R}{rupees} = \text{Monthly revenue generated in rupees}.
\]	
\paragraph{Operational Cost: How much is the operations cost of the emporium?}
The score for operational cost is given as,
\begin{align*}
	\tsub{Score}{op. cost} &= 0, \tsubb{OC}{rupees} \ge \text{Threshold} \\
	&= \sqrt{\dfrac{\text{Threshold}-\tsubb{OC}{rupees}}{\text{Threshold}}}, \text{Otherwise}.
\end{align*}
where,
\[
	\tsubb{OC}{rupees} = \text{operating cost in rupees}.
\]	

The overall score for Business Specific is given as,
\begin{align*}
	\tsub{Score}{business specific} &= \tsubb{W}{revenue} \times \tsub{Score}{revenue} \\
	&+ \tsubb{W}{op. cost} \times \tsub{Score}{op. cost}
\end{align*}

\subsubsection{Customers}
\paragraph{No. of customers: How many customers do you get in a month?}
The score for number of customers in a month is given as,
\begin{align*}
	\tsub{Score}{no. of custom.} &= 1, \tsubb{N}{customers} \ge \text{Threshold} \\
	&= \Big(\dfrac{\tsubb{N}{customers}}{\text{Threshold}}\Big).
\end{align*}
where,
\[
	\tsubb{N}{customers} = \text{No. of customers in a month.}
\]
\paragraph{Customer type: Who among the above selected are you major customers?}
The options given for customer type are,
\begin{ttable}
	Customer type & Rank \\ \hline
	Customers within the state & 1 \\
	Other traders & 2 \\
	Tourists from other Indian states & 3 \\
	Foreign Tourists & 4 \\ \hline
	Sum of ranks & 10
\end{ttable}
The score is given as,
\multsel{customer type}

The overall score for customers is given as,
\begin{align*}
	\tsub{Score}{customers} &= \tsubb{W}{no. of custom.} \times \tsub{Score}{no. of custom.} \\
	&+ \tsubb{W}{customer type} \times \tsub{Score}{customer type}
\end{align*}
\subsubsection{Operations}
\paragraph{Suppliers: Whom among the following do you procure the finished products from?}
The options given are,
\begin{ttable}
	Supplier & Rank \\ \hline
	Individual artisans or weaver & 1 \\
	Clusters of rural artisans and villagers & 2 \\
	Societies and SHG groups & 3 \\
	Craft centers & 4 \\
	Common facility center & 5 \\
	Manufacturing & 6 \\ \hline
	Sum of ranks & 21
\end{ttable}
The score of suppliers is given as,
\multsel{suppliers}
\paragraph{Procurement: Do you procure products from artisans in bulk?}
Now depending on the answer, the score is given as,
\begin{align*}
	\tsub{Score}{procurement} &= 0, \text{If no} \\
	&= 1, \text{If yes}.
\end{align*}
\paragraph{Order placement: How can orders be placed for the products in the emporium?}
The options given are,
\begin{ttable}
	Order placement & Rank \\ \hline
	Others & 1 \\
	Direct orders placed in the emporium & 2 \\
	Through phone calls & 3 \\
	Online requests & 4 \\ \hline
	Sum of ranks. & 10
\end{ttable}
The score is calculated as,
\multsel{order placement}
\paragraph{Payment mode: Which mode of payment do you accept?}
The options given are,
\begin{ttable}
	Payment mode & Rank \\ \hline
	Others & 1 \\
	Cash Payment & 2 \\
	Debit and Credit card & 3 \\
	Online payments & 4 \\ \hline
	Sum of ranks. & 10
\end{ttable}
The score is calculated as,
\multsel{payment mode}
\paragraph{Accessibility: How is the products from the artisans brought to the emporium?}
The options given are,
\begin{ttable}
	Transport for acquiring & Rank \\ \hline
	Artisans transport the product & 1 \\
	Emporiums provides transportation for the products & 2 \\ \hline
	Sum of ranks & 3
\end{ttable}
The score is calculated as,
\multsel{transp. acq.}
\paragraph{Accessibility: How many outlets do you have in town and cities?}
The score is calculated based on number of outlets as,
\begin{align*}
	\tsub{Score}{outlets} &= 1, \tsubb{N}{outlets} \ge \text{Threshold} \\
	&= \sqrt{\dfrac{\tsubb{N}{outlets}}{\text{Threshold}}}, \text{Otherwise}.
\end{align*}
where,
\[
	\tsubb{N}{outlets} = \text{No. of outlets}.
\]
The accessibility score is calculated as,
\begin{align*}
	\tsub{Score}{accessibility} &= \tsubb{W}{transp. acq.} \times \tsub{Score}{transp. acq.} \\
	&+ \tsubb{W}{outlets} \times \tsub{Score}{outlets}.
\end{align*}

The overall score for 'Operations' is given as,
\begin{align*}
	\tsub{Score}{operations} &= \tsubb{W}{suppliers} \times \tsub{Score}{suppliers} \\
	&+ \tsubb{W}{order placement} \times \tsub{Score}{order placement} \\
	&+ \tsubb{W}{payment mode} \times \tsub{Score}{payment mode} \\
	&+ \tsubb{W}{accessibility} \times \tsub{Score}{accessibility} \\
	&+ \tsubb{W}{procurement} \times \tsub{Score}{procurement}
\end{align*}
\subsubsection{Market potential}
\paragraph{Product reach: Are your products being displayed at exhibitions and expos?}
The score is given as,
\begin{align*}
	\tsub{Score}{product reach} &= 0, \text{If no} \\
	&= 1, \text{If yes}.
\end{align*}
\paragraph{Market potential: Do you get bulk orders for you products?}
the score is given as,
\begin{align*}
	\tsub{Score}{bulk orders} &= 0, \text{if no} \\
	&= 1, \text{if yes}.
\end{align*}
\paragraph{Market potential: Are the products sold in your emporium branded?}
the score is given as,
\begin{align*}
	\tsub{Score}{branded} &= 0, \text{if no} \\
	&= 1, \text{if yes}.
\end{align*}

The overall score for market potential is given as,
\begin{align*}
	\tsub{Score}{market potential} &= \tsubb{W}{reach} \times \tsub{Score}{reach} \\
	&+ \tsubb{W}{bulk orders} \times \tsub{Score}{bulk orders} \\
	&+ \tsubb{W}{branded} \times \tsub{Score}{branded}
\end{align*}
\subsubsection{Customer services}
\paragraph{Services: Do you provide delivery services for your products?}
The score is given as,
\begin{align*}
	\tsub{Score}{delivery} &= 0, \text{if no} \\
	&= 1, \text{if yes}.
\end{align*}
\paragraph{Catalog: Do you have a catalog of your products?}
the score is given as,
\begin{align*}
	\tsub{Score}{catalog} &= 0, \text{if no} \\
	&= 1, \text{if yes}.
\end{align*}

The overall score for 'Customer Services' is given as,
\begin{align*}
	\tsub{Score}{customer services} &= \tsubb{W}{delivery} \times \tsub{Score}{delivery} \\
	&+ \tsubb{W}{catalog} \times \tsub{Score}{catalog}
\end{align*}
\end{document}

