\documentclass[oneside,twocolumn]{article}
\usepackage{mathtools}

%% ADD PACKAGES %%

\title{Attributes tagging for Craft Center}
\author{Kipa Tachak}
\date{Thursday  6 June 2019}

\setcounter{secnumdepth}{3}
\newcommand{\tsub}[2]{\text{#1}_{\text{#2}}}
\newcommand{\tsubb}[2]{#1_{\text{#2}}}
\newcommand{\dsub}[2]{\dfrac{\text{#1}}{\text{#2}}}
\newcommand{\multsel}[1]
{
	\[
		\tsub{Score}{#1} = \dsub{Sum of selected}{Sum of ranks}
	\]
}
\newcommand{\singsel}[1]
{
	\[
		\tsub{Score}{#1} = \dsub{Rank of selected}{Max. rank}.
	\]
}

\newenvironment{ttable}
{
\begin{center}
\begin{tabular}{c|c}
\hline
}
{
\\ \hline
\end{tabular}
\end{center}
}

\begin{document}
\maketitle

%% BODY %%
\section{About the document}
This file documents the process of asset tagging for the asset 'Craft Center' in 'Hand loom and Handicrafts'.

\section{Attributes and questions}
The sections Observation and General are similar to other assets. Hence we will be only discussing the specific section in this document.

The list of variables and questions are given as:
\begin{itemize}
	\item Initial Observation.
	\item General
	\item Specific
		\begin{itemize}
			\item Basic Info
				\begin{enumerate}
					\item Description: Which of the following training is provided in the center?
					\item Number of seats: How many seats are available in the craft centers?
					\item Students enrolled: How many students are enrolled in the craft center?
					\item Trainers employed: How many trainers are employed in the craft center?
					\item Duration: What are the duration of the courses you offer?
					\item Asset potential: Which courses has the maximum number of enrollments?
				\end{enumerate}
			\item Asset Utilization
				\begin{enumerate}
					\item Course completion: What percentage of enrolled students successfully complete the course?
					\item Professional undertaking: What proportion of the passed-out students take the training into the professional front?
				\end{enumerate}
			\item Operations
				\begin{enumerate}
					\item Incentive for students: What incentives are provided to the students enrolled in the center?
					\item Product market: Where are the finished products sold?
					\item Payment mode: Which mode of payment do you accept?
				\end{enumerate}
			\item Marketing and advertisement
				\begin{enumerate}
					\item Means of advertisement: How are the courses advertised to prospective trainees?
				\end{enumerate}
			\item Market potential
				\begin{enumerate}
					\item Market potential: Are the products sold in your emporium branded?
				\end{enumerate}
			\item Customer services
				\begin{enumerate}
					\item Catalog: Do you have a cataloger of your products?
				\end{enumerate}
		\end{itemize}
\end{itemize}

\section{Overall scoring}
The overall scoring of the asset is given as,
\begin{align*}
	\tsub{Score}{asset} &= \tsubb{W}{observation} \times \tsub{Score}{observation} \\
	&+ \tsubb{W}{general} \times \tsub{Score}{general} \\
	&+ \tsubb{W}{specific} \times \tsub{Score}{specific}.
\end{align*}

The overall scoring of each variable is done as,
\[
	\tsub{Score}{var.} = \sum_{i=1}^{n} \tsubb{W}{i} \times \tsub{Score}{sub param.}
\]
\section{Attribute scoring and the process}
\subsection{Specific}
\subsubsection{Basic Info}
\paragraph{Description: Which of the following training is provided in the center?} 
The options given are,
\begin{ttable}
	Training provided & Rank \\ \hline
	Knitting & 1 \\
	Cane and Bamboo & 2 \\
	Carpet making & 3 \\
	Weaving & 4 \\
	Tailoring & 5 \\
	Carpentry & 6 \\
	Monpa & 7 \\
	Bell metal & 8 \\
	Black Smiting & 9 \\
	Jewelry making & 10 \\ \hline
	Sum of ranks & 48 
\end{ttable}
The score is calculated as,
\multsel{description}
\paragraph{Number of seats: How many seats are available in the craft centers?}
The score is given as,
\begin{align*}
	\tsub{Score}{seats} &= 1, \tsubb{N}{seats} \ge \text{Threshold} \\
	&= \sqrt{\dfrac{\tsubb{N}{seats}}{\text{Threshold}}}, \text{Otherwise}
\end{align*}
\paragraph{Students enrolled: How many students are enrolled in the craft center?}
The score is given as,
\begin{align*}
	\tsub{Score}{enrollment} &= 1, \tsubb{N}{enrolls} \ge \text{Threshold} \\
	&= \sqrt{\dfrac{\tsubb{N}{enrolls}}{\text{Threshold}}}, \text{Otherwise}
\end{align*}
\paragraph{Trainers employed: How many trainers are employed in the craft center?}
The score is given as,
\begin{align*}
	\tsub{Score}{trainers} &= 1, \tsubb{N}{trainers} \ge \text{Threshold} \\
	&= \sqrt{\dfrac{\tsubb{N}{trainers}}{\text{Threshold}}}, \text{Otherwise}
\end{align*}
\paragraph{Duration: What are the duration of the courses you offer?}
The options are given as,
\begin{ttable}
	Duration & Ranks \\ \hline
	< 1 year & 1 \\
	1 year & 2 \\
	2 years & 3 \\
	3 years & 4 \\
	4 years & 5 \\
	5 years & 6 \\ \hline
	Sum of ranks & 21
\end{ttable}
\multsel{duration}

\paragraph{Asset potential: Which courses has the maximum number of enrollments?}
The options given are,
\begin{ttable}
	Training provided & Rank \\ \hline
	Knitting & 1 \\
	Cane and Bamboo & 2 \\
	Carpet making & 3 \\
	Weaving & 4 \\
	Tailoring & 5 \\
	Carpentry & 6 \\
	Monpa & 7 \\
	Bell metal & 8 \\
	Black Smiting & 9 \\
	Jewelry making & 10 \\ \hline
	Sum of ranks & 48 
\end{ttable}
The score is calculated as,
\multsel{asset potential}

\subsubsection{Asset Utilization}
\paragraph{Course completion: What percentage of enrolled students successfully complete the course?}
The score is given as,
\[
	\tsub{Score}{completion} = \dfrac{\% \text{completion}}{100}.
\]	
\paragraph{Professional undertaking: What proportion of the passed-out students take the training into the professional front?}
The available options are,
\begin{ttable}
	Percentage & Rank \\ \hline
	0-20\% & 1 \\
	20-40\% & 2 \\
	40-60\% & 3 \\
	60-80\% & 4 \\
	80-100\% & 5 \\ \hline
	Sum of ranks & 15
\end{ttable}
The score is given as,
\singsel{prof. under.}

\subsubsection{Operations}
\paragraph{Incentives for students: What incentives are provided to the students enrolled in the center?}
The options given are,
\begin{ttable}
	Incentives & Rank \\ \hline
	None & 0 \\
	Stipend & 1 \\
	Certification & 2 \\
	Placement opportunities & 3 \\
	Scholarships & 4 \\ \hline
	Sum of ranks & 10
\end{ttable}
The score of incentives is given as,
\multsel{incentives}
\paragraph{Product market: Where are the finished products sold?}
The options given as,
\begin{ttable}
	Market options & Ranks \\ \hline
	Sold directly from the craft center & 1 \\
	Sold to emporium & 2 \\
	Sold to retail shops & 3 \\
	Sold to wholesale markets & 4 \\
	Sold in exhibitions/expos & 5 \\ \hline
	Sum of ranks & 15
\end{ttable}
The score is given as,
\multsel{prod. market}
\paragraph{Payment mode: Which mode of payment do you accept?}
The options given are,
\begin{ttable}
	Payment mode & Rank \\ \hline
	Others & 1 \\
	Cash Payment & 2 \\
	Debit and Credit card & 3 \\
	Online payments & 4 \\ \hline
	Sum of ranks. & 10
\end{ttable}
The score is calculated as,
\multsel{payment mode}
\subsubsection{Marketing and advertisement}
\paragraph{Means of advertisement: How are the courses advertised to prospective trainees?}
The options given are,
\begin{ttable}
	Advertisement options & Rank \\ \hline
	Others & 1 \\
	Word of mouth & 2 \\
	Social media & 3 \\
	Newspaper and print & 4 \\
	Govt. websites & 5 \\ \hline
	Sum of ranks & 15
\end{ttable}
The score is given as,
\multsel{advertisement}
\subsubsection{Market potential}
\paragraph{Branding: Are the products sold in your emporium branded?}
The score is given as,
\begin{align*}
	\tsub{Score}{branded} &= 0, \text{if no} \\
	&= 1, \text{if yes}.
\end{align*}

\subsubsection{Customer services}
\paragraph{Catalog: Do you have a catalog of your products?}
the score is given as,
\begin{align*}
	\tsub{Score}{catalog} &= 0, \text{if no} \\
	&= 1, \text{if yes}.
\end{align*}

\end{document}

