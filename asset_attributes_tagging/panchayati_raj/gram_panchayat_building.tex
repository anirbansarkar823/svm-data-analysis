\documentclass[oneside,twocolumn]{article}
\usepackage{mathtools}

%% ADD PACKAGES %%

\title{Attributes tagging for Gram Panchayat Building}
\author{Kipa Tachak}
\date{Saturday  8 June 2019}

\setcounter{secnumdepth}{3}
\newcommand{\tsub}[2]{\text{#1}_{\text{#2}}}
\newcommand{\tsubb}[2]{#1_{\text{#2}}}
\newcommand{\dsub}[2]{\dfrac{\text{#1}}{\text{#2}}}
\newcommand{\multsel}[1]
{
	\[
		\tsub{Score}{#1} = \dsub{Sum of selected}{Sum of ranks}
	\]
}
\newcommand{\singsel}[1]
{
	\[
		\tsub{Score}{#1} = \dsub{Rank of selected}{Max. rank}.
	\]
}
\newenvironment{ttable}
{
\begin{center}
\begin{tabular}{c|c}
\hline
}
{
\\ \hline
\end{tabular}
\end{center}
}

\begin{document}
\maketitle

%% BODY %%
\section{About the document}
This file documents the process of asset tagging for the asset 'Gram Panchayat Building' in the vertical 'Panchayati Raj'.

\section{Attributes and questions}
The list of sections, variables, and questions are:
\begin{itemize}
	\item Observation
	\item General
	\item Asset Specific
		\begin{itemize}
			\item Infrastructure
				\begin{enumerate}
					\item Utility: Is the building being utilized currently?
				\end{enumerate}
			\item Services
				\begin{enumerate}
					\item Availability:  What kinds of services are available there?
				\end{enumerate}
			\item Records Maintenance
				\begin{enumerate}
					\item Records: Is the village data maintained in the building?
				\end{enumerate}
			\item Frequency
				\begin{enumerate}
					\item Efficiency: How frequently are the gram sabhas held?
				\end{enumerate}
			\item Community Outreach
				\begin{enumerate}
					\item Reach: How many people attend these gram sabha meetings on an average?
				\end{enumerate}
			\item Communication
				\begin{enumerate}
					\item Mass Gathering: How does the panchayat pass information to the community?
				\end{enumerate}
		\end{itemize}
\end{itemize}

\section{Overall scoring}
The overall scoring of the asset is given as,
\begin{align*}
	\tsub{Score}{asset} &= \tsubb{W}{observation} \times \tsub{Score}{observation} \\
	&+ \tsubb{W}{general} \times \tsub{Score}{general} \\
	&+ \tsubb{W}{specific} \times \tsub{Score}{specific}.
\end{align*}

The score of a section or a variable is given as,
\[
	\tsub{Score}{var} = \sum_{l \in P} \tsubb{W}{l} \times \tsub{Score}{l}.
\]
where,
\[
	P = \text{Set of sub-parameter labels.}
\]	

\section{Attribute scoring and the process}
\subsection{Asset Specific}
\subsubsection{Infrastructure}
\paragraph{Utility: Is the building being utilized currently?}
The score is given as,
\begin{align*}
	\tsub{Score}{utility} &= 0, \text{if no} \\
	&= 1, \text{if yes}.
\end{align*}
\subsubsection{Services}
\paragraph{Availability:  What kinds of services are available there?}
The options given are,
\begin{ttable}
	Services & Rank \\ \hline
	None & 0 \\
	Application Schemes & 1 \\
	Aadhar Enrollment & 2 \\
	Both & 3 \\ \hline
	Max. rank & 3
\end{ttable}
The score is given as,
\singsel{services}
\subsubsection{Records Maintenance}
\paragraph{Records: Is the village data maintained in the building?}
The options given are,
\begin{ttable}
	Maintenance & Rank \\ \hline
	Not maintained & 0 \\
	Manually recorded & 1 \\
	Digitized & 2 \\ \hline
	Max. rank & 2
\end{ttable}
The score is given as,
\singsel{records}
\subsubsection{Frequency}
\paragraph{Efficiency: How frequently are the gram sabhas held?}
The options given are,
\begin{ttable}
	Frequency & Rank \\ \hline
	Whenever required & 1 \\
	Yearly & 2 \\
	Quarterly & 3 \\
	Once in two months & 4 \\
	At least once a month & 5 \\ \hline
	Max. rank & 5
\end{ttable}
The score is given as,
\singsel{frequency}
\subsubsection{Community Outreach}
\paragraph{Reach: How many people attend these gram sabha meetings on an average?}
The score is given as,
\begin{align*}
	\tsub{Score}{reach} &= 1, \tsubb{N}{attendees} \ge \text{Threshold} \\
	&= \Big( \dfrac{\tsubb{N}{attendees}}{\text{Threshold}} \Big)^{2}, \text{Otherwise}.
\end{align*}
\subsubsection{Communication}
\paragraph{Mass Gathering: How does the panchayat pass information to the community?}
The options given are,
\begin{ttable}
	Sharing options & Rank \\ \hline
	Others & 1 \\
	In Panchayat meetings & 2 \\
	Door to door & 3 \\
	Pamphlets & 4 \\
	Phone calls & 5 \\
	Messages & 6 \\ \hline
	Sum of ranks & 21
\end{ttable}
The score is calculated as,
\multsel{communication}

\end{document}

