\documentclass[oneside,twocolumn]{article}
\usepackage{mathtools}

%% ADD PACKAGES %%

\title{Attributes tagging for Bore wells.}
\author{Kipa Tachak}
\date{Thursday  6 June 2019}

\newcommand{\tsub}[2]{\text{#1}_{\text{#2}}}
\newcommand{\tsubb}[2]{#1_{\text{#2}}}
\newcommand{\dsub}[2]{\dfrac{\text{#1}}{\text{#2}}}
\newenvironment{ttable}
{
\begin{center}
\begin{tabular}{c|c}
\hline
}
{
\\ \hline
\end{tabular}
\end{center}
}

\begin{document}
\maketitle

%% BODY %%
\section{About the document}
This file documents the process of asset tagging for the asset
'Bore wells' in the category 'Water and
Sanitation'.
\section{Attributes and questions}
There are three sections 'Observation', 'General', and 'Specific'
, each containing their own list of variables and sub-parameters.
For each sub-parameter, there is associated a question for
gathering data related to that sub-parameter.

Since, the structure and to some extend the content of questions
in Observation and General are similar to other assets, we will
analyze from these sections only the variables which differ
significantly from that in other assets. As always, in-depth analysis of section 'Specific' will be given.

The list of variables and questions that will be analyzed is given below:
\begin{itemize}
	\item Observation
	\item General
	\item Specific:
	\begin{enumerate}
		\item Security: What is the material that is used to case the Bore well?
		\item Connection Type: The Bore well is connected to which of the following assets?
		\item Reliability: How much time will it take for the water to come up after switching on the motor?
		\item Caution: What is the depth of the bore well that you can observe?
		\item Cross connections: What are the assets that you can observe near the bore well?
		\item Capacity: How much liter per day can be provided at maximum?
		\item Condition: Were people drinking water from this bore well while you visited this spot?
		\item Quality: Would you drink the water from this bore well?
		\item Charges: Are there any charges which come when getting water from this bore well?
		\item Maintenance: What is the frequency of technical maintenance or safety audit? 
	\end{enumerate}
\end{itemize}

\section{Overall scoring}
The overall scoring of the asset is given as:
\begin{align*}
	\tsub{Score}{asset} &= \tsubb{W}{observation} \times \tsub{Score}{observation} \\
	&+\tsubb{W}{general} \times \tsub{Score}{general} \\
	&+ \tsub{W}{specific} \times \tsub{Score}{specific}
\end{align*}
\section{Attribute scoring and the process}
\subsection{Specific}
\subsubsection{Security}
\paragraph{What is the material that is used to case the bore well?}
The options given are,
\begin{ttable}
	Material & Rank \\ \hline
	None & 0 \\
	Others & 1 \\
	Cement & 2 \\
	PVC & 3 \\
	Steel & 4 \\ \hline
	Max. rank & 4 
\end{ttable}
The score is given as,
\[
	\tsub{Score}{security} = \dsub{Rank of selected}{Max. Rank}
\]

\subsubsection{Connection Type}
\paragraph{The bore well is connected to which of the following assets?}
This is a multi-choice multi-select question. The options are given as,
\begin{ttable}
	Connected Asset & Rank \\ \hline
	None & 0 \\
	Others & 1 \\
	Hand pump & 2 \\
	Storage tank & 3 \\
	Over ground tank & 4 \\
	Under ground tank & 5 \\
	Over head tank & 6 \\ \hline
	Max. rank & 6
\end{ttable}
The score is given as,
\[
	\tsub{Score}{connect. type} = \dsub{Rank of selected}{Max. Rank}
\]

\subsubsection{Reliability}
\paragraph{How much time will it take for the water to come up after switching on the motor?}
Classified as 'Confusing'. Suitable scoring technique have to be chosen depending on the input.
\textbf{Suggestion} The input should be given as numeric values representing time taken by the water to reach the surface in seconds.

\subsubsection{Caution}
\paragraph{What is the depth of the bore well that you can observe?}
The options given are,
\begin{ttable}
	Depth & Rank \\ \hline
	30-50m & 4 \\
	51-70m & 3 \\
	91-120m & 2 \\
	121-150m & 1 \\
	Above 150m & 0 \\ \hline
	Max. rank & 4
\end{ttable}
Since, the more deep a bore well is, the more dangerous and requirement for caution. Therefore, more dangerous bore wells will have lower score. The scoring is given as,
\[
	\tsub{Score}{caution} = \dsub{Rank of selected}{Max. rank}.
\]
\subsubsection{Cross connections}
\paragraph{What are the assets that you can observe near the bore well?}
The options given are,
\begin{ttable}
	Assets & Rank \\ \hline
	Waste Disposal Ground & 0 \\
	Fertilizer Storage & 1 \\
	Septic Tank & 2 \\
	Burial Ground & 3 \\
	Livestock yard & 4 \\
	None & 5 \\ \hline
	Sum of ranks & 15
\end{ttable}

The score is calculated as,
\[
	\tsub{Score}{cross conn.} = \dsub{Sum of selected options}{Sum of ranks}
\]

\subsubsection{Capacity}
\paragraph{How much liters per day can be provided at maximum?}
The input to this question is numeric (capacity in liters). The score is calculated as,
\begin{align*}
	\tsub{Score}{capacity} &= 1, \tsubb{V}{water} \ge \text{Threshold} \\
	&= \Big( \dfrac{\tsubb{V}{water}}{\text{Threshold}} \Big)^{2}, \text{Otherwise}.
\end{align*}
where,
\[
	\tsubb{V}{water} = \text{Max. volume of water per day (liters)}.
\]	
\subsubsection{Condition}
\paragraph{Were people drinking water from this bore well while you visited this spot?}
\begin{align*}
	\tsub{Score}{condition} &= 0, \text{If no} \\
	&= 1, \text{if yes}.
\end{align*}

\subsubsection{Quality}
\paragraph{Would you drink the water from this bore well?}
\begin{align*}
	\tsub{Score}{quality} &= 0, \text{If no} \\
	&= 1, \text{if yes}.
\end{align*}

\subsubsection{Charges}
\paragraph{Are there any charges which come when getting water from this bore well?}
The presence/absence of charges is given as,
\begin{align*}
	\text{Pr}{charges} &= 0, \text{If no} \\
	&= 1, \text{if yes}.
\end{align*}
Then, the score for water charges is given as,
\begin{align*}
	\tsub{Score}{charges} &= 0, \tsubb{Ch}{water} \ge \text{Threshold} \\
	&= 1 - \Big( \dfrac{\tsubb{Ch}{water}}{\text{Threshold}} \Big), \text{Otherwise}.
\end{align*}
where,
\[
	\tsubb{Ch}{water} = \text{Charges on water in rupees}.
\]
\subsubsection{Maintenance}
\paragraph{What is the frequency of technical maintenance or safety audit?}
The options for technical maintenance are,
\begin{ttable}
	Maintenance Frequency & Rank \\ \hline
	Never & 0 \\
	Others & 1 \\
	Irregular & 2 \\
	Yearly & 3 \\
	Monthly & 4 \\ \hline
	Max. rank & 4 
\end{ttable}
The score for technical maintenance is,
\[
	\tsub{Score}{technical maint.} = \dsub{Rank of selected}{Max. Rank}.
\]

The options for safety audit are,
\begin{ttable}
	Audit Frequency & Rank \\ \hline
	Never & 0 \\
	Others & 1 \\
	Irregular & 2 \\
	Yearly & 3 \\
	Monthly & 4 \\ \hline
	Max. rank & 4 
\end{ttable}
The score for safety audit is given as,
\[
	\tsub{Score}{safety audit} = \dsub{Rank of selected}{Max. Rank}.
\]

The overall score for maintenance is,
\begin{align*}
	\tsub{Score}{maintenance} &= \tsubb{W}{technical maint.} \times \tsub{Score}{technical maint.} \\
	&+ \tsubb{W}{safety audit} \times \tsub{Score}{safety audit}.
\end{align*}

\end{document}

