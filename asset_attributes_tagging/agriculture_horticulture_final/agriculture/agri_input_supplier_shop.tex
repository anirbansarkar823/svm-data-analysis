\documentclass[oneside,twocolumn]{article}
\usepackage{mathtools}

\setcounter{secnumdepth}{0}

\title{Attributes tagging for Agriculture Inputs shop/Supplier.}
\author{Kipa Tachak}
\date{Tuesday 4 June 2019}

\newcommand{\tsub}[2]{\text{#1}_{\text{#2}}}
\newcommand{\tsubb}[2]{#1_{\text{#2}}}
\newcommand{\dsub}[2]{\dfrac{\text{#1}}{\text{#2}}}
\newenvironment{ttable}
               {
                 \begin{center}
                   \begin{tabular}{c|c}
                     \hline
               }
               {
                 \\ \hline
                   \end{tabular}
                 \end{center}
               }

               \begin{document}
               \maketitle

               \section{About the document}
               This file documents the process of asset tagging for the asset
               'Agriculture Inputs Shop/Supplier' in agriculture and horticulture.
               \section{Variables and questions}
               There are three sections Observation, General, and Specific. Most of
               the questions in Observation and General are similar to other
               assets. So, we will only analyze the parts where there is some
               change. Hence, the list of questions in each section to be
               analyzed is given below.
               \begin{itemize}
               \item Observation
               \item General
               \item Specific
                 \begin{enumerate}
                 \item Where do you procure inputs required for the facility?
                 \item What are different chemical inputs available in the facility?
                 \item What are different organic inputs available in the facility?
                 \item Where do you store the agri inputs sold in the shop?
                 \item How many cubic meters are available for storage?
                 \item How does the facility advertise about the services available?
                 \item What is the average operational cost per month to run
                   the facility? (Excluding salary of the employees)
                 \item What is the total average income per month generated from the facility?
                 \item How many types of inputs are available in the
                   facility?
                 \item What are the five major type of inputs purchased
                   by the customers from the facility?
                 \item What is the average cost of the top 5 most purchased goods in facility?(In Rupees)
                 \item Which type of customers purchase the inputs from
                   the facility?
                 \item How many customers purchase the inputs from the facility in a year?
                 \item How do the customers transport goods from the facility?
                 \end{enumerate}
               \end{itemize}

               \section{Overall scoring}
               The overall scoring of the asset is given  as,
               \begin{align*}
                 \tsub{Score}{asset} &= \tsubb{W}{observation} \times \tsub{Score}{observation} \\
                 &+ \tsubb{W}{general} \times \tsub{Score}{general} \\
                 &+ \tsubb{W}{specific} \times \tsub{Score}{specific}.
               \end{align*}
               
               \section{Scoring and the process}
               \subsection{Specific}
               \subsubsection{Where do you procure inputs required for the facility?}
               The available options for procuring inputs are,
               \begin{ttable}
                 Sources & Rank \\ \hline
                 None & 0 \\
                 Others & 1 \\
                 Local supplier & 2 \\
                 Supplier in other states & 3 \\
                 Research organization & 4 \\
                 Govt. agency & 5 \\ \hline
                 Sum of ranks & 15
               \end{ttable}

               The score is calculated as,
               \[
               \tsub{Score}{input source} = \dfrac{\text{sum of selected sources}}{\text{sum of ranks}}.
               \]

               \subsubsection{What are different chemical inputs available in the facility?}
               The options for chemical inputs are,
               \begin{ttable}
                 Chemical Inputs & Rank \\ \hline
                 None & 0 \\
                 Others & 1 \\
                 Fertilizers & 2 \\
                 Pesticides & 3 \\
                 Growth hormones & 4 \\ \hline
                 Sum of ranks & 10
               \end{ttable}

               The score is calculated as,
               \[
               \tsub{Score}{chemical inputs} = \dfrac{\text{Sum of selected chemicals}}{\text{Sum of ranks}}.
               \]

               \subsubsection{What are different organic inputs available in the facility?}
               The options for organic inputs are,
               \begin{ttable}
                 Organic Inputs & Rank \\ \hline
                 None & 0 \\
                 Others & 1 \\
                 Fertilizers & 2 \\
                 Pesticides & 3 \\
                 Growth hormones & 4 \\ \hline
                 Sum of ranks & 10
               \end{ttable}

               The score is calculated as,
               \[
               \tsub{Score}{organic inputs} = \dfrac{\text{Sum of selected chemicals}}{\text{Sum of ranks}}.
               \]

               \subsubsection{Where do you store the agri inputs sold in the shop?}
               This variable denotes the storage options used. The question asked is a
multi-choice multi-select type of question. The options with ranks are
given below.
\begin{center}
  \begin{tabular}{c|c}
    \hline
    Storage Options & Rank \\ \hline
    None & 0 \\
    Others & 1 \\
    Home & 2 \\
    Within the shop & 3 \\
    Private Godown/Warehouse & 4 \\ \hline
    Sum of ranks & 10 \\ \hline
  \end{tabular}
\end{center}
The scoring of storage mode is given as.
\[
score_{storage mode} = \dfrac{\text{sum of selected options}}{\text{sum of ranks}}.
\]
 \subsubsection{How many cubic meters are available for storage?}
               The score for volume of storage is given as,
               \begin{align*}
                 \tsub{Score}{volume} &= 1, V_{storage} \ge thresh \\
                 &= 10 \times \sqrt{\dfrac{V_{storage}}{thresh}}, otherwise.
               \end{align*}

               Now, depending on whether the storage is water proofed or not
               we set,
               \[
               \tsub{Present}{water proofing} = 0 \text{ or } 1.
               \]
               Then the score of storage is given as,
               \begin{align*}
                 \tsub{Score}{storage} &= 0.80 \times \tsub{Score}{volume} \\
                 &+ 0.20 \times \tsub{Present}{water proofing}.
               \end{align*}
               
               \subsubsection{How does the facility advertise about the services available?}
               The available advertisement methods are,
               \begin{ttable}
                 Advertisement Mode & Rank \\ \hline
                 Posters/Display boards & 1 \\
                 Notice & 2 \\
                 Village Meeting & 3 \\
                 Announcements/Broadcast & 4 \\
                 Marketing team/Sales & 5 \\ \hline
                 Sum of ranks & 15
               \end{ttable}

               The score is calculated as,
               \[
               \tsub{Score}{advertisement} = \dfrac{\text{Sum of selected modes}}{\text{Sum of ranks}}.
               \]

               \subsubsection{What is the average operational cost per month to run the facility?(Excluding Salary of the Employees)}
               The score for operational cost is calculated as,
               \begin{align*}
                 \tsub{Score}{Op. cost} &= 0, C_{op} > thresh \\
                 &= 1, C_{op} \le 0 \\
                 &= \Big(\dfrac{thresh - C_{op}}{thresh}\Big)^2, \text{otherwise}
               \end{align*}
               where,
               \begin{align*}
                 C_{op} &= \text{Avg. operational cost}. \\
                 thresh &= \text{Threshold Op. cost}.
               \end{align*}

               \subsubsection{What is the total average income per month generated from the facility?}
               The score for average income per month generated from the facility is given
               as,
               \begin{align*}
                 \tsub{Score}{avg. income} &= 1, I_{per month} \ge threshold \\
                 &= \Big(\dfrac{I_{per month}}{threshold}\Big)^{2}.
               \end{align*}
               where,
               \[
               I_{per month} = \text{Avg. Income generated per month.}
               \]
               \subsubsection{How many types of inputs are available in the facility?}
               The score for the amount of input types available in the facility is
               given as,
               \begin{align*}
                 \tsub{Score}{input types} &= 1, C_{input types} \ge thresh \\
                 &= 10 \times \sqrt{\dfrac{C_{input types}}{thresh}}, \text{otherwise}
               \end{align*}
               where,
               \begin{align*}
                 C_{input types} &= \text{Type count of inputs.} \\
                 thresh &= \text{Threshold}
               \end{align*}

               \subsubsection{What are the 5 major type of inputs purchased by the customers from the facility?}
               Classified as question for 'Information and Clarity'.

               \subsubsection{What is the average cost of the top 5 most purchased goods in facility?(In Rupees)}
               The options for payment bill are given as,
               \begin{ttable}
                 Payment bill mode & Rank \\ \hline
                 None & 0 \\
                 Others & 1 \\
                 Hand Written & 2 \\
                 Computerized Bill & 3 \\ \hline
                 Sum of ranks & 6
               \end{ttable}
               
               The score for the question is calculated as,
               \begin{align*}
                 \tsub{Score}{avg. cost} &= 0 &, c > thresh \\
                 &= 1 &, c \le 0 \\
                 &= 10 \times \sqrt{\dfrac{thresh - c}{thresh}} &, \text{otherwise}.
               \end{align*}
               where,
               \begin{align*}
                 c &= \text{avg. cost of 5 most purchased goods}. \\
                 thresh &= \text{threshold cost}.
               \end{align*}

               The score for payment bill mode is given as,
               \[
               \tsub{Score}{payment bill mode} = \dsub{sum of selected methods}{sum of ranks}.
               \]

               Then, the combined score for average cost for 5 most purchased goods and
               payment bill mode is given as,
               \begin{align*}
                 \tsub{Score}{avg. cost\&bill mode} &= 0.90 \times \tsub{Score}{avg. cost} \\
                 &+ 0.10 \times \tsub{Score}{bill mode}.
               \end{align*}

               \subsubsection{Which type of customers purchase the inputs from the facility?}
               The options for customer types are,
               \begin{ttable}
                 Customer types & Rank \\ \hline
                 None & 0 \\
                 Others & 1 \\
                 Farmers & 2 \\
                 NGO & 3 \\
                 Research Organizations & 4 \\
                 Govt. Agencies & 5 \\ \hline
                 Sum of ranks & 15
               \end{ttable}
               The score is calculated as,
               \[
               \tsub{Score}{customer type} = \dsub{Sum of selected types}{Sum of ranks}.
               \]

               \subsubsection{How many customers purchase the inputs from the facility in a year?}
               The score is calculated as,
               \begin{align*}
                 \tsub{Score}{no. of customer} &= 1, \text{no. of customer} \ge \text{Threshold} \\
                 &= 10 \times \sqrt{\dsub{no. of customers}{Threshold}}, \text{otherwise}.
               \end{align*}

               \subsubsection{How do the customers transport goods from the facility?}
               The options for transport are,
               \begin{ttable}
                 Transport Mode & Rank \\ \hline
                 Head Load & 1 \\
                 Two Wheeler & 2 \\
                 Tempo/Auto & 3 \\
                 Sumo/Tracker & 4 \\
                 Mini Luggage Transport Vehicle & 5 \\
                 Huge Luggage Transport Vehicle & 6 \\ \hline
                 Sum of ranks & 21
               \end{ttable}
               The scoring is done as,
               \[
               \tsub{Score}{customer tranp.} = \dsub{Sum of selected modes}{Sum of ranks}
               \]
               \end{document}
