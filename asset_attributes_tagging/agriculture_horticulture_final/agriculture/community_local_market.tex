\documentclass[oneside,twocolumn]{article}
\usepackage{mathtools}

%% ADD PACKAGES %%

\title{Attributes tagging for Community Local Market.}
\author{Kipa Tachak}
\date{Tuesday  4 June 2019}

\newcommand{\tsub}[2]{\text{#1}_{\text{#2}}}
\newcommand{\tsubb}[2]{#1_{\text{#2}}}
\newcommand{\dsub}[2]{\dfrac{\text{#1}}{\text{#2}}}
\newenvironment{ttable}
               {\begin{center}
                 \begin{tabular}{c|c}
                   \hline
                   }
                   { \\ \hline
                 \end{tabular}
               \end{center}
               }

\begin{document}
\maketitle

%% BODY %%

\section{About the document}
This file documents the process of asset tagging for the asset
'Community Local Market' in agriculture and horticulture.

\section{Variables and questions}
There are three sections Observation, General, and Specific. Most of
the questions in Observation and General are similar to other
assets. So, we will only analyze the parts where there is some
change. Hence, the list of questions in each section to be
analyzed is given below.
\begin{itemize}
\item Observation
\item General
\item Specific
  \begin{enumerate}
  \item Where do you procure inputs required for the market?
  \item Where do you store the produce sold in the shops?
  \item How many cubic meters are available for storage?
  \item Do you have a cold storage facility?
  \item How often did you need to throw away agricultural produce last week?
  \item How do you document wastage of food?
  \item How does the market advertise about the services available?
  \item What is the average operational cost per month to run the stall?(Excluding Salary of the Employees)
  \item What is the average income per month from each stall?
  \item What kind of permission is required to setup a stall in the market?
  \item What is the payment mechanism used in the market?
  \item What is the weighing mechanism used in the market?
  \item How many types of products are available in the market?
  \item What are the 5 major type of products purchased by the customers from the market?
  \item What is the average cost of the top 5 most purchased goods in market?(In Rupees)
  \item How does sellers determine the selling price of the products?
  \item Which type of customers purchase the inputs from the market?
  \item How many customers purchase the inputs from the market in a year?
  \item How do the customers transport goods from the market?
  \end{enumerate}
\end{itemize}

\section{Overall scoring}
The overall scoring of the asset is given  as,
\begin{align*}
  \tsub{Score}{asset} &= \tsubb{W}{observation} \times \tsub{Score}{observation} \\
  &+ \tsubb{W}{general} \times \tsub{Score}{general} \\
  &+ \tsubb{W}{specific} \times \tsub{Score}{specific}.
\end{align*}

\section{Scoring and the process}
\subsection{Specific}
\subsubsection{Where do you procure inputs required for the market?}
The available options for procuring inputs are,
\begin{ttable}
  Sources & Rank \\ \hline
  None & 0 \\
  Others & 1 \\
  Local supplier & 2 \\
  Supplier in other states & 3 \\
  Research organization & 4 \\
  Govt. agency & 5 \\ \hline
  Sum of ranks & 15
\end{ttable}

The score is calculated as,
\[
\tsub{Score}{input source} = \dfrac{\text{sum of selected sources}}{\text{sum of ranks}}.
\]

\subsubsection{Where do you store the produce sold in the shops?}
This variable denotes the storage options used. The question asked is a
multi-choice multi-select type of question. The options with ranks are
given below.
\begin{center}
  \begin{tabular}{c|c}
    \hline
    Storage Options & Rank \\ \hline
    None & 0 \\
    Others & 1 \\
    Home & 2 \\
    Within the shop & 3 \\
    Private Godown/Warehouse & 4 \\ \hline
    Sum of ranks & 10 \\ \hline
  \end{tabular}
\end{center}
The scoring of storage mode is given as.
\[
score_{storage mode} = \dfrac{\text{sum of selected options}}{\text{sum of ranks}}.
\]

\subsubsection{How many cubic meters are available for storage?}
The score for volume of storage is given as,
\begin{align*}
  \tsub{Score}{volume} &= 1, V_{storage} \ge thresh \\
  &= \sqrt{\dfrac{V_{storage}}{thresh}}, otherwise.
\end{align*}

Now, depending on whether the storage is water proofed or not
we set,
\[
\tsub{Present}{water proofing} = 0 \text{ or } 1.
\]
Then the score of storage is given as,
\begin{align*}
  \tsub{Score}{storage} &= 0.80 \times \tsub{Score}{volume} \\
  &+ 0.20 \times \tsub{Present}{water proofing}.
\end{align*}

\subsubsection{Do you have a cold storage facility?}
For, availability of cold storage, we have,
\begin{align*}
  \text{available} =& 0 &, \text{ if cold storage not available.} \\
  =& 1 &, \text{ if cold storage available.}
\end{align*}

If cold storage is available we will be given its capacity. Hence, the score is given by,
\[
\text{Score}_{\text{cold storage}} = \text{available} \times \text{Score}_{\text{volume}}.
\]
where,
\[
\text{Score}_{\text{Volume}} = \dfrac{\text{Volume in } m^{3}}{\text{Threshold}}.
\]

\subsubsection{How often did you need to throw away agricultural produce last week?}
Score for wastage last week is calculated as,
\[
\text{Score}_{\text{wastage}} = \dfrac{(\text{Threshold} - \text{Frequency of wastage last week})^2}{\text{Threshold}^2}.
\]

\subsubsection{How do you document wastage of food?}
The options for wastage documentation are given in the table.
\begin{center}
  \begin{tabular}{c | c}
    \hline
    Documentation mode & Rank \\ \hline
    None & 0 \\
    Others & 1 \\
    Book keeping & 2 \\
    Digital tools (offline) & 3 \\
    Digital tools (online) & 4 \\ \hline
    sum of ranks & 10 \\ \hline
  \end{tabular}
\end{center}

The score is calculated as,
\[
\text{Score}_{\text{documentation}} = \dfrac{\text{sum of selected modes}}{\text{sum of ranks}}.
\]

\subsubsection{How does the market advertise about the services available?}
The available advertisement methods are,
\begin{ttable}
  Advertisement Mode & Rank \\ \hline
  Posters/Display boards & 1 \\
  Notice & 2 \\
  Village Meeting & 3 \\
  Announcements/Broadcast & 4 \\
  Marketing team/Sales & 5 \\ \hline
  Sum of ranks & 15
\end{ttable}

The score is calculated as,
\[
\tsub{Score}{advertisement} = \dfrac{\text{Sum of selected modes}}{\text{Sum of ranks}}.
\]

\subsubsection{What is the average operational cost per month to run the
  stall?(Excluding Salary of the Employees)}
The score for operational cost is calculated as,
\begin{align*}
  \tsub{Score}{Op. cost} &= 0, C_{op} > thresh \\
  &= 1, C_{op} \le 0 \\
  &= \Big(\dfrac{thresh - C_{op}}{thresh}\Big)^2, \text{otherwise}
\end{align*}
where,
\begin{align*}
  C_{op} &= \text{Avg. operational cost}. \\
  thresh &= \text{Threshold Op. cost}.
\end{align*}

\subsubsection{What is the average income per month from each stall?}
The score for average income per month generated from each stall is given
as,
\begin{align*}
  \tsub{Score}{avg. income} &= 1, I_{per month} \ge threshold \\
  &= \Big(\dfrac{I_{per month}}{threshold}\Big)^{2}.
\end{align*}
where,
\[
I_{per month} = \text{Avg. Income generated per month.}
\]

\subsubsection{What kind of permission is required to setup a stall in the market?}
The ranks of various permissions options are given below.
\begin{center}
  \begin{tabular}{c | c}
    \hline
    Permitting bodies & Rank \\ \hline
    None & 0 \\
    Others & 1 \\
    Permission from the market & 2 \\
    Permission from village head/community & 3 \\
    Permission from Govt. Agency & 4 \\ \hline
    Sum of ranks & 10 \\ \hline
  \end{tabular}
\end{center}

The scoring for permission is done as,
\[
\text{Score}_{\text{Permission}} = \dfrac{\text{Sum of selected bodies}}{\text{Sum of ranks}}.
\]

\subsubsection{What is the payment mechanism used in the market?}
The payment options along with their ranks are given
below.
\begin{center}
  \begin{tabular}{c|c}
    \hline
    Payment method & Rank \\ \hline
    Cash & 1 \\
    Others & 2 \\
    POS Machine & 3 \\
    Digital Payment(PayTM, BHIM) & 4 \\ \hline
    Sum of Ranks & 10 \\ \hline
  \end{tabular}
\end{center}

The scoring is done as,
\[
\text{Score}_{\text{Payment mechanism}} = \dfrac{\text{sum of selected methods}}{\text{Sum of ranks}}.
\]

\subsubsection{What is the weighing mechanism used in the market?}
The weighing options are given below with their ranks.
\begin{center}
  \begin{tabular}{c|c}
    \hline
    Weighing method & Rank \\ \hline
    None & 0 \\
    Mechanical scale & 1 \\
    Electronic scale & 2 \\ \hline
    Sum of ranks & 3 \\ \hline
  \end{tabular}
\end{center}

The score is calculated as,
\[
\text{Score}_{\text{Weight mechanism}} = \dfrac{\text{sum of selected methods}}{\text{sum of ranks}}.
\]

\subsubsection{How many types of products are available in the market?}
The score for the amount of product types available in the market is
given as,
\begin{align*}
  \tsub{Score}{product types} &= 1, C_{product types} \ge thresh \\
  &= \sqrt{\dfrac{C_{product types}}{thresh}}, \text{otherwise}
\end{align*}
where,
\begin{align*}
  C_{product types} &= \text{Type count of products.} \\
  thresh &= \text{Threshold}
\end{align*}

\subsubsection{What are the 5 major type of products purchased by the customers from the market?}
Classified as question for 'Information and Clarity'.

\subsubsection{What is the average cost of the top 5 most purchased goods in market?(In Rupees)}
The options for payment bill are given as,
\begin{ttable}
  Payment bill mode & Rank \\ \hline
  None & 0 \\
  Others & 1 \\
  Hand Written & 2 \\
  Computerized Bill & 3 \\ \hline
  Sum of ranks & 6
\end{ttable}

The score for the question is calculated as,
\begin{align*}
  \tsub{Score}{avg. cost} &= 0 &, c > thresh \\
  &= 1 &, c \le 0 \\
  &= \sqrt{\dfrac{thresh - c}{thresh}} &, \text{otherwise}.
\end{align*}
where,
\begin{align*}
  c &= \text{avg. cost of 5 most purchased goods}. \\
  thresh &= \text{threshold cost}.
\end{align*}
The score for payment bill mode is given as,
\[
\tsub{Score}{payment bill mode} = \dsub{sum of selected methods}{sum of ranks}.
\]
Then, the combined score for average cost for 5 most purchased goods and
payment bill mode is given as,
\begin{align*}
  \tsub{Score}{avg. cost\&bill mode} &= 0.90 \times \tsub{Score}{avg. cost} \\
  &+ 0.10 \times \tsub{Score}{bill mode}.
\end{align*}

\subsubsection{How does sellers determine the selling price of the products?}
The options for price determination are given below.
\begin{center}
  \begin{tabular}{c|c}
    \hline
    Modes & Ranks \\ \hline
    other & 1 \\
    based on market price & 2 \\
    depended on buyers & 3 \\
    bargaining & 4 \\
    government fixed & 5 \\ \hline
    sum of ranks & 15 \\ \hline
  \end{tabular}
\end{center}
The score is calculated as,
\[
\text{Score}_{\text{Cost Policy}} = \dfrac{\text{sum of selected price modes}}{\text{sum of ranks}}.
\]

\subsubsection{Which type of customers purchase the inputs from the market?}
The options for customer types are,
\begin{ttable}
  Customer types & Rank \\ \hline
  None & 0 \\
  Others & 1 \\
  Farmers & 2 \\
  NGO & 3 \\
  Research Organizations & 4 \\
  Govt. Agencies & 5 \\ \hline
  Sum of ranks & 15
\end{ttable}
The score is calculated as,
\[
\tsub{Score}{customer type} = \dsub{Sum of selected types}{Sum of ranks}.
\]

\subsubsection{How many customers purchase the inputs from the market in a year?}
The score is calculated as,
\begin{align*}
  \tsub{Score}{no. of customer} &= 1, \text{no. of customer} \ge \text{Threshold} \\
  &= \sqrt{\dsub{no. of customers}{Threshold}}, \text{otherwise}.
\end{align*}

\subsubsection{How do the customers transport goods from the market?}
The options for transport are,
\begin{ttable}
  Transport Mode & Rank \\ \hline
  Head Load & 1 \\
  Two Wheeler & 2 \\
  Tempo/Auto & 3 \\
  Sumo/Tracker & 4 \\
  Mini Luggage Transport Vehicle & 5 \\
  Huge Luggage Transport Vehicle & 6 \\ \hline
  Sum of ranks & 21
\end{ttable}
The scoring is done as,
\[
\tsub{Score}{customer tranp.} = \dsub{Sum of selected modes}{Sum of ranks}
\]
\end{document}

