\documentclass[oneside,twocolumn]{article}
\usepackage{mathtools}

\setcounter{secnumdepth}{0}

\title{Attributes tagging for Plant/Seed Unit.}
\author{Kipa Tachak}
\date{Tuesday 4 June 2019}

\newcommand{\tsub}[2]{\text{#1}_{\text{#2}}}
\newcommand{\tsubb}[2]{#1_{\text{#2}}}
\newcommand{\dsub}[2]{\dfrac{\text{#1}}{\text{#2}}}
\newenvironment{ttable}
               {
                 \begin{center}
                   \begin{tabular}{c|c}
                     \hline
               }
               {
                 \\ \hline
                   \end{tabular}
                 \end{center}
               }

               \begin{document}
               \maketitle

               \section{About the document}
               This file documents the process of asset tagging for the asset
               'Plant/Seed Unit' in agriculture and horticulture.
               \section{Variables and questions}
               There are three sections Observation, General, and Specific. Most of
               the questions in Observation and General are similar to other
               assets. So, we will only analyze the parts where there is some
               change. Hence, the list of questions in each section to be
               analyzed is given below.
               \begin{itemize}
               \item Observation
               \item General
               \item Specific
                 \begin{enumerate}
                 \item Where do you procure inputs required for the soil lab?
                 \item What are different chemical inputs used in the unit?
                 \item What type(s) of green house present in the unit?
                 \item What is the water source used for irrigation mechanism?
                 \item What is the type of irrigation mechanism used in the unit?
                 \item How does the unit advertise about the services
                   available?
                 \item What is the average operational cost per month to run
                   the unit? (Excluding salary of the employees)
                 \item How may square meters are available for storage of the
                   seeds/plants produced?
                 \item How many types of plants/seeds are available in the
                   unit?
                 \item What are the five major type of plants/seeds purchased
                   by the customers from the unit?
                 \item Are the seeds/plants sold to the customers organic?
                 \item What is the average cost per plant/per sack of seed in
                   the unit? (in rupees)
                 \item Which type of customers purchase the plants/seeds from
                   the unit?
                 \item How many customers purchase the plants/seeds from the
                   unit in a year?
                 \item How does the customers transport the goods purchased
                   from the unit?
                 \end{enumerate}
               \end{itemize}

               \section{Overall scoring}
               The overall scoring of the asset is given  as,
               \begin{align*}
                 \tsub{Score}{asset} &= \tsubb{W}{observation} \times \tsub{Score}{observation} \\
                 &+ \tsubb{W}{general} \times \tsub{Score}{general} \\
                 &+ \tsubb{W}{specific} \times \tsub{Score}{specific}.
               \end{align*}
               
               \section{Scoring and the process}
               \subsection{Specific}
               \subsubsection{Where do you procure inputs required for the unit?}
               The available options for procuring inputs are,
               \begin{ttable}
                 Sources & Rank \\ \hline
                 None & 0 \\
                 Others & 1 \\
                 Local supplier & 2 \\
                 Supplier in other states & 3 \\
                 Research organization & 4 \\
                 Govt. agency & 5 \\ \hline
                 Sum of ranks & 15
               \end{ttable}

               The score is calculated as,
               \[
               \tsub{Score}{input source} = \dfrac{\text{sum of selected sources}}{\text{sum of ranks}}.
               \]

               \subsubsection{What are different chemical inputs used in the unit?}
               The options for chemical inputs are,
               \begin{ttable}
                 Chemical Inputs & Rank \\ \hline
                 None & 0 \\
                 Others & 1 \\
                 Fertilizers & 2 \\
                 Pesticides & 3 \\
                 Growth hormones & 4 \\ \hline
                 Sum of ranks & 10
               \end{ttable}

               The score is calculated as,
               \[
               \tsub{Score}{chemical inputs} = \dfrac{\text{Sum of selected chemicals}}{\text{Sum of ranks}}.
               \]

               \subsubsection{What type(s) of green house present in the unit?}
               The options for green house are,
               \begin{ttable}
                 Green house type & Rank \\ \hline
                 None & 0 \\
                 Others & 1 \\
                 Bamboo house & 2 \\
                 Lath house & 3 \\
                 Poly house & 4 \\
                 Mesh house & 5 \\
                 Glass house & 6 \\ \hline
                 Sum of ranks & 21
               \end{ttable}

               The score is calculated as,
               \[
               \tsub{Score}{green house} = \dsub{Sum of selected types}{Sum of ranks}.
               \]

               \subsubsection{What is the water source used for irrigation mechanism?}
               The options for irrigation source are,
               \begin{ttable}
                 Irrigation Source & Rank \\ \hline
                 Others & 1 \\
                 Rain fed & 2 \\
                 Lake/Pond & 3 \\
                 Ground & 4 \\
                 River/Stream & 5 \\ \hline
                 Sum of ranks & 15
               \end{ttable}

               The score of irrigation source is calculated as,
               \[
               \tsub{Score}{irrig. source} = \dsub{Sum of selected sources}{Sum of ranks}.
               \]

               \subsubsection{What is the type of irrigation mechanism used in the unit?}
               The options for irrigation mechanisms are,
               \begin{ttable}
                 Irrigation Mechanism & Rank \\ \hline
                 Others & 1 \\
                 Rose Can & 2 \\
                 Sprinklers & 3 \\
                 Drip Irrigation & 4 \\
                 Canal Irrigation & 5 \\ \hline
                 Sum of ranks & 15
               \end{ttable}

               The score is calculated as,
               \[
               \tsub{Score}{irrig. mechanism} = \dsub{Sum of selected mechanism}{Sum of ranks}.
               \]
               
               \subsubsection{How does the unit advertise about the services available?}
               The available advertisement methods are,
               \begin{ttable}
                 Advertisement Mode & Rank \\ \hline
                 Posters/Display boards & 1 \\
                 Notice & 2 \\
                 Village Meeting & 3 \\
                 Announcements/Broadcast & 4 \\
                 Marketing team/Sales & 5 \\ \hline
                 Sum of ranks & 15
               \end{ttable}

               The score is calculated as,
               \[
               \tsub{Score}{advertisement} = \dfrac{\text{Sum of selected modes}}{\text{Sum of ranks}}.
               \]

               \subsubsection{What is the average operational cost per month to run the unit? (Excluding salary of the employees)}
               The score for operational cost is calculated as,
               \begin{align*}
                 \tsub{Score}{Op. cost} &= 0, C_{op} > thresh \\
                 &= 1, C_{op} \le 0 \\
                 &= \Big(\dfrac{thresh - C_{op}}{thresh}\Big)^2, \text{otherwise}
               \end{align*}
               where,
               \begin{align*}
                 C_{op} &= \text{Avg. operational cost}. \\
                 thresh &= \text{Threshold Op. cost}.
               \end{align*}

               \subsubsection{How many square meters are available for storage of the seeds/plants produced?}
               The score for volume of storage is given as,
               \begin{align*}
                 \tsub{Score}{volume} &= 1, V_{storage} \ge thresh \\
                 &= 10 \times \sqrt{\dfrac{V_{storage}}{thresh}}, otherwise.
               \end{align*}

               Now, depending on whether the storage is water proofed or not
               we set,
               \[
               \tsub{Present}{water proofing} = 0 \text{ or } 1.
               \]
               Then the score of storage is given as,
               \begin{align*}
                 \tsub{Score}{storage} &= 0.80 \times \tsub{Score}{volume} \\
                 &+ 0.20 \times \tsub{Present}{water proofing}.
               \end{align*}

               \subsubsection{How many types of plants/seeds are available in the unit?}
               The score for the amount of plants/seeds available in the unit is
               given as,
               \begin{align*}
                 \tsub{Score}{plants/seeds} &= 1, C_{plant/seeds} \ge thresh \\
                 &= 10 \times \sqrt{\dfrac{C_{plant/seeds}}{thresh}}, \text{otherwise}
               \end{align*}
               where,
               \begin{align*}
                 C_{plants/seeds} &= \text{Type count of plants/seeds.} \\
                 thresh &= \text{Threshold}
               \end{align*}

               \subsubsection{What are the five major type of plants/seeds purchased by the customers from the unit?}
               Classified as question for 'Information and Clarity'.

               \subsubsection{Are the seeds/plants sold to the customers organic?}
               The score is calculated as,
               \[
               \tsub{Score}{organic} = 0.5 \times yes_{organic} + 0.5 \times Pr_{certification}
               \]
               where,
               \begin{align*}
                 yes_{organic} &= 0 \text{ or } 1 \\
                 Pr_{certification} &= \text{0 or 1 depending on certificate presence.}
               \end{align*}

               \subsubsection{What is the average cost per plant/per pack of seed in the unit? (in rupees)}
               The options for payment mechanism are given as,
               \begin{ttable}
                 Payment mechanism & Rank \\ \hline
                 Others & 1 \\
                 Cash & 2 \\
                 POS Machine & 3 \\
                 Mobile Apps (Paytm, BHIM) & 4 \\ \hline
                 Sum of ranks & 10
               \end{ttable}

               The score for the question is calculated as,
               \begin{align*}
                 \tsub{Score}{cost per nos.} &= 0 &, c > thresh \\
                 &= 1 &, c \le 0 \\
                 &= 10 \times \sqrt{\dfrac{thresh - c}{thresh}} &, \text{otherwise}.
               \end{align*}
               where,
               \begin{align*}
                 c &= \text{avg. cost per plant/per pack seeds in rupees}. \\
                 thresh &= \text{threshold cost}.
               \end{align*}

               The score for payment method is given as,
               \[
               \tsub{Score}{payment method} = \dsub{sum of selected methods}{sum of ranks}.
               \]

               Then, the combined score for cost per plant/per pack seeds is given as,
               \begin{align*}
                 \tsub{Score}{cost plants/seeds} &= 0.80 \times \tsub{Score}{cost} \\
                 &+ 0.20 \times \tsub{Score}{payment}.
               \end{align*}

               \subsubsection{Which type of customers purchase the plants/seeds from the unit?}
               The options for customer types are,
               \begin{ttable}
                 Customer types & Rank \\ \hline
                 None & 0 \\
                 Others & 1 \\
                 Farmers & 2 \\
                 NGO & 3 \\
                 Research Organizations & 4 \\
                 Govt. Agencies & 5 \\ \hline
                 Sum of ranks & 15
               \end{ttable}
               The score is calculated as,
               \[
               \tsub{Score}{customer type} = \dsub{Sum of selected types}{Sum of ranks}.
               \]

               \subsubsection{How many customers purchase the plants/seeds from the unit in a year?}
               The score is calculated as,
               \begin{align*}
                 \tsub{Score}{no. of customer} &= 1, \text{no. of customer} \ge \text{Threshold} \\
                 &= 10 \times \sqrt{\dsub{no. of customers}{Threshold}}, \text{otherwise}.
               \end{align*}

               \subsubsection{How does the customers transport the goods purchased from the unit?}
               The options for transport are,
               \begin{ttable}
                 Transport Mode & Rank \\ \hline
                 Head Load & 1 \\
                 Two Wheeler & 2 \\
                 Tempo/Auto & 3 \\
                 Sumo/Tracker & 4 \\
                 Mini Luggage Transport Vehicle & 5 \\
                 Huge Luggage Transport Vehicle & 6 \\ \hline
                 Sum of ranks & 21
               \end{ttable}
               The scoring is done as,
               \[
               \tsub{Score}{customer tranp.} = \dsub{Sum of selected modes}{Sum of ranks}
               \]
               \end{document}
