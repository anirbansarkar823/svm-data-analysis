\documentclass[oneside,twocolumn]{article}
\usepackage{mathtools}

%% ADD PACKAGES %%

\title{Attributes tagging for Fish Pond.}
\author{Kipa Tachak}
\date{Monday  3 June 2019}

\setcounter{secnumdepth}{0}
\newcommand{\tsub}[2]{\text{#1}_{\text{#2}}}
\newenvironment{ttable}
               {\begin{center}
                   \begin{tabular}{c|c}
                     \hline
               }
               {\end{tabular}
                 \end{center}
               }
               
\begin{document}
\maketitle

%% BODY %%
\section{About the document}
In this document, we analyze the attributes of the asset 'Fish Pond' and assign suitable scoring criteria based on the attributes.
\section{Attributes}
The attributes of the asset are divided into three sections, Observation,
General, and Specific. The methods for scoring Observation, and General,
are same for other assets. Hence, in this document, we will only discuss
the specific section.

A tentative list of variables and sub-parameters for Specific section
is given below;
\begin{itemize}
  \item Inputs:
    \begin{itemize}
    \item Procurement
    \item Chemical Inputs
    \item Organic Inputs
    \end{itemize}
  \item Irrigation
    \begin{itemize}
      \item Type
    \end{itemize}
  \item Marketing
    \begin{itemize}
      \item Advertisement
    \end{itemize}
  \item Products
    \begin{itemize}
    \item Quantity
    \item Type
    \item Cost
    \item Cost Policy
    \end{itemize}
  \item Best Practices
    \begin{itemize}
    \item Inter-cropping
    \item Farm Equipment
    \end{itemize}
  \item Customers
    \begin{itemize}
    \item Type
    \end{itemize}
  \item Logistics
    \begin{itemize}
    \item Transport of goods
    \end{itemize}
  \item Inventory
    \begin{itemize}
    \item Inventory Capacity
    \item Cold Storage
    \end{itemize}
  \item Food Wastage
    \begin{itemize}
    \item Wastage
    \item Documentation
    \end{itemize}
  \item Market
    \begin{itemize}
    \item Buyer
    \end{itemize}
\end{itemize}
\section{Overall scoring}
The overall scoring of the asset is done according to the following equation;
\begin{align*}
  score_{\text{asset}} &= W_{\text{Observation}} \times score_{\text{Observation}} \\
  &+ W_{\text{General}} \times score_{\text{General}} \\
  &+ W_{\text{Specific}} \times score_{\text{Specific}}.
\end{align*}
\section{Attributes scoring and the process}
The overall score for the specific section is given as,
\begin{align*}
  \text{Score}_{\text{specific}} &= W_{\text{Inputs}} \times \text{Score}_{\text{Inputs}}
  \\
  &+ W_{\text{Irrigation}} \times \text{Score}_{\text{Irrigation}} \\
  &+ \dots + W_{\text{Market}} \times \text{Score}_{\text{Market}}.
\end{align*}

\subsection{Inputs}
This variable denotes the strategy used in availing inputs for the asset. It has three sub-parameters, Procurement, Chemical Inputs, and Organic Inputs. The overall
score of the variable is given as,
\begin{align*}
  \text{Score}_{\text{Inputs}} &= W_{\text{procurement}} \times \text{Score}_{\text{procurement}} \\
  &+ W_{\text{chemical inputs}} \times \text{Score}_{\text{chemical inputs}} \\
  &+ W_{\text{organic inputs}} \times \text{Score}_{\text{organic inputs}}.
\end{align*}

\subsubsection{Procurement}
The available places for procurement are given with their
respective rank in the table below.
\begin{center}
  \begin{tabular}{c | c}
    \hline
    Procurement Options & Rank \\ \hline
    None & 0 \\
    others & 1 \\
    Supplier in other states & 2 \\
    Local Supplier & 3 \\
    Research Organization & 4 \\
    Government Agency & 5 \\ \hline
    Sum of ranks & 15 \\ \hline
  \end{tabular}
\end{center}
The score for inputs is calculated as,
\begin{align*}
  score_{\text{procurement}} = \dfrac{\text{sum of selected options}}{\text{sum of ranks}}.
\end{align*}

\paragraph{Note}
The score is relative i.e., it only shows one set of options is better or
worse than other. However, it does not show by how much the goodness or badness differs.

\subsubsection{Chemical Inputs}
This sub-parameter scores the asset based on its chemical inputs. The options available for the sub-parameter are given below,
\begin{ttable}
  Chemical Input types & Rank \\ \hline
  None & 0 \\
  Others & 1 \\
  Fertilizer & 2 \\
  Pesticide & 2 \\
  Growth Hormones & 3 \\ \hline
  Sum of ranks & 8 \\ \hline
\end{ttable}
The score is calculated as,
\[
\tsub{Score}{chemical inputs} = \dfrac{\text{sum of selected chemicals}}{\text{Sum of ranks}}.
\]

\subsubsection{Organic Inputs}
This sub-parameter scores the asset based on its organic inputs. The options available for the sub-parameter are given below,
\begin{ttable}
  Organic Input types & Rank \\ \hline
  None & 0 \\
  Others & 1 \\
  Fertilizer & 2 \\
  Pesticide & 2 \\
  Growth Hormones & 3 \\ \hline
  Sum of ranks & 8 \\ \hline
\end{ttable}
The score is calculated as,
\[
\tsub{Score}{organic inputs} = \dfrac{\text{sum of selected organics}}{\text{Sum of ranks}}.
\]
\subsection{Irrigation}
This variable scores the asset based on the type of irrigation present. The options present for irrigation are,
\begin{ttable}
  Irrigation Type & Ranks \\ \hline
  Others & 1 \\
  Rose can & 2 \\
  Sprinklers & 3 \\
  Drip Irrigation & 4 \\
  Canal Irrigation & 5 \\ \hline
  Sum of ranks & 15 \\ \hline
\end{ttable}
The scoring is done as follows.
\[
\tsub{Score}{Irrigation} = \dfrac{\text{sum of selected Irrig. methods}}{\text{Sum of ranks}}.
\]
\subsection{Marketing}
This variable is about the advertisement options available. The available
options with ranks are given below.
\begin{center}
  \begin{tabular}{c|c}
    \hline
    Advertisement options & Ranks \\ \hline
    None & 0 \\
    Others & 1 \\
    Village meetings & 2 \\
    Announcements/Broadcasts & 3 \\
    Posters/Display boards & 4 \\
    Multimedia & 5 \\
    Marketing team & 6 \\
    Sales agent & 7 \\ \hline
    Sum of ranks & 28 \\ \hline
  \end{tabular}
\end{center}
The score is calculated as,
\[
\text{score}_{\text{Marketing}} = \dfrac{\text{Sum of selected options}}{\text{sum of ranks}}.
\]

\subsection{Products}
The variable products measures the overall product diversity and quality from the asset.
The overall scoring is done as follows,
\begin{align*}
  \text{Score}_{\text{products}} &= W_{\text{Quantity}} \times \text{Score}_{\text{Quantity}} \\
  &= W_{\text{Cost}} \times \text{Score}_{\text{Cost}} \\
  &= W_{\text{Cost Policy}} \times \text{Score}_{\text{Cost policy}}.
\end{align*}

\subsubsection{Quantity}
The sub-parameters scores the asset on the basis of the amount of produce in a year from the asset. The score calculation is as follows,
\begin{align*}
  \tsub{Score}{Quantity} &= 1 &, \text{produce} > \text{Threshold}. \\
  &= \sqrt{\dfrac{\text{produce}}{\text{Threshold}}} &, \text{produce} \le \text{Threshold}.
\end{align*}
where, \text{produce} is in kilograms.

\subsubsection{Type}
We have avoided including the variable Type for scoring. We regard this variable as used for 'information and clarity'.

\subsubsection{Cost}
This variable is used for evaluating the average cost per unit of the products produced from the asset. The score is calculated as,
\begin{align*}
  \text{Score}_{\text{Cost}} &= 0 &, \text{avg. cost} > \text{Threshold} \\
  &= \sqrt{\dfrac{\text{Threshold} - \text{avg. cost}}{\text{Threshold}}} &, \text{avg. cost} \le \text{Threshold}.
\end{align*}
Where, \text{avg. cost} is average cost per unit of the products produced.

\subsubsection{Cost Policy}
Sub-parameter for evaluating the modes of determining the selling price of the products. The available modes are given below.
\begin{center}
  \begin{tabular}{c|c}
    \hline
    Modes & Ranks \\ \hline
    other & 1 \\
    based on market price & 2 \\
    depended on buyers & 3 \\
    bargaining & 4 \\
    government fixed & 5 \\ \hline
    sum of ranks & 15 \\ \hline
  \end{tabular}
\end{center}
The score is calculated as,
\[
\text{Score}_{\text{Cost Policy}} = \dfrac{\text{sum of selected price modes}}{\text{sum of ranks}}.
\]

\subsection{Best Practices}
This variable scores the asset based on the good practices used in maintaining the asset.
The score for best practices is calculated as,
\begin{align*}
  \tsub{Score}{best practices} &= 0.25 \times \text{Inter-cropping used} \\
  &+ 0.75 \times \tsub{Score}{farm equipment}.
\end{align*}

\subsubsection{Inter-cropping}
This sub-parameter denotes the use of inter-cropping technique and is given as,
\begin{align*}
\text{Inter-cropping used} &= 0 &, \text{if inter-cropping not used}. \\
&= 1 &, \text{if inter-cropping used}.
\end{align*}

\subsubsection{Farm Equipment}
Available options for Farm equipment are given as,
\begin{ttable}
  Farm Equipment Option & Rank \\ \hline
  None & 0 \\
  Other & 1 \\
  Ploughing & 2 \\
  Sowing & 2 \\
  Spraying Inputs & 3 \\
  Removing Unwanted crops & 2 \\
  Harvesting & 2 \\
  Transportation & 2 \\ \hline
  sum of ranks & 14 \\ \hline
\end{ttable}
The score is calculated as,
\[
\tsub{Score}{farm equipment} = \dfrac{\text{sum of selected farm equipment}}{\text{sum of ranks}}.
\]
\subsection{Customers}
The scoring for customers is done as,
\[
  \text{Score}_{\text{customers}} = \text{Score}_{\text{Type}}.
\]

\subsubsection{Type}
Evaluate the type of customers. The available customers types are,
\begin{center}
  \begin{tabular}{c|c}
    \hline
    Customer type & Rank \\ \hline
    None & 0 \\
    Others & 1 \\
    Farmers & 2 \\
    NGOs & 3 \\
    Research Orgs. & 4 \\
    Govt. Agencies & 5 \\ \hline
    Sum of ranks & 15 \\ \hline
  \end{tabular}
\end{center}
The score is calculated as,
\[
\text{Score}_{\text{type}} = \dfrac{\text{sum of selected types}}{\text{sum of ranks}}.
\]

\subsection{Logistics}
The variable has only one sub-parameter 'Transport of goods'. The available options for transporting goods are given below.
\begin{center}
  \begin{tabular}{c | c}
    \hline
    Transport Option & Rank \\ \hline
    Head load & 1 \\
    Two wheeler & 2 \\
    Tempo/Auto & 3 \\
    Sumo/Tracker & 4 \\
    Mini Luggage Transport Vehicle & 5 \\
    Huge Luggage Transport Vehicle & 6 \\ \hline
    Sum of ranks & 21 \\ \hline
  \end{tabular}
\end{center}

The score for logistics is calculated as,
\[
\text{Score}_{\text{Logistics}} = \dfrac{\text{sum of selected transports}}{\text{sum of ranks}}.
\]

\subsection{Inventory}
The variable has two sub-parameters, Inventory capacity and Cold storage. The scoring of Inventory variable is given as,
\begin{align*}
  \text{Score}_{\text{Inventory}} &= W_{\text{capacity}} \times \text{Score}_{\text{capacity}} \\
  &+ W_{\text{cold storage}} \times \text{Score}_{\text{cold storage}}.
\end{align*}

\subsubsection{Inventory Capacity}
If the inventory is water proof, we give the score as calculated from its capacity. If the inventory is partially water proof, we will deduct 5\% from the score and if water proofing is unavailable we will deduct 10\% from the score.

Now depending on whether water proofing is available or not we have,
\begin{align*}
  \text{water proofing} &= 0 &, \text{if fully water proof.} \\
  &= 0.05 &, \text{if partially water proof.} \\
  &= 0.10 &, \text{if no water proofing.}
\end{align*}

Hence, the score of inventory capacity is calculated as,
\[
\text{Score}_{\text{capacity}} = \text{Score}_{\text{Volume}} ( \dfrac{100 - \text{water proofing}}{100} )
\]
where,
\[
\text{Score}_{\text{Volume}} = \dfrac{\text{Volume in } m^{3}}{\text{Threshold}}.
\]

\subsubsection{Cold Storage}
For, availability of cold storage, we have,
\begin{align*}
  \text{available} =& 0 &, \text{ if cold storage not available.} \\
  =& 1 &, \text{ if cold storage available.}
\end{align*}

If cold storage is available we will be given its capacity. Hence, the score is given by,
\[
\text{Score}_{\text{cold storage}} = \text{available} \times \text{Score}_{\text{volume}}.
\]
where,
\[
\text{Score}_{\text{Volume}} = \dfrac{\text{Volume in } m^{3}}{\text{Threshold}}.
\]

\subsection{Food Wastage}
The variable has two sub-parameters for evaluating wastage of food. The score is given as,
\begin{align*}
\text{Score}_{\text{food wastage}} &= W_{\text{wastage}} \times \text{Score}_{\text{wastage}} \\
&+ W_{\text{documentation}} \times \text{Score}_{\text{documentation}}.
\end{align*}
\subsubsection{Wastage}
Score for wastage is calculated as,
\[
\text{Score}_{\text{wastage}} = \dfrac{(\text{Threshold} - \text{Frequency of wastage})^2}{\text{Threshold}^2}.
\]

\subsubsection{Documentation}
The options for wastage documentation are given in the table.
\begin{center}
  \begin{tabular}{c | c}
    \hline
    Documentation mode & Rank \\ \hline
    None & 0 \\
    Others & 1 \\
    Book keeping & 2 \\
    Digital tools (offline) & 3 \\
    Digital tools (online) & 4 \\ \hline
    sum of ranks & 10 \\ \hline
  \end{tabular}
\end{center}
The score is calculated as,
\[
\text{Score}_{\text{documentation}} = \dfrac{\text{sum of selected modes}}{\text{sum of ranks}}.
\]

\subsection{Market}
This variable scores the asset based on its access to market. It has one sub-parameter
Buyer. The score is given as,
\[
\tsub{Score}{market} = \tsub{Score}{Buyer}.
\]

\subsubsection{Buyer}
Buyer has to sub-question, one for the type of buyer and another for the quantity of buyer.
\paragraph{Buyer Type}
The available type of buyers are given below.
\begin{ttable}
  Buyer type & Rank \\ \hline
  Private & 1 \\
  Govt. & 2 \\ \hline
  sum of ranks & 3 \\ \hline
\end{ttable}
The score for buyer type is calculated as,
\[
\tsub{Score}{buyer type} = \dfrac{\text{sum of selected types}}{\text{sum of ranks}}.
\]

\paragraph{Number of buyers}
The score for number of buyers is given as,
\begin{align*}
  \tsub{Score}{No. of buyers} =& 1 &, \text{if buyers} > \text{threshold}. \\
  =& \sqrt{\dfrac{\text{Number of buyers}}{Threshold}} &, \text{if buyers} \leq \text{threshold}.
\end{align*}

Then, the score of Buyer is given as,
\[
\tsub{Score}{buyer} = 0.25 \times \tsub{Score}{type} + 0.75 \times \tsub{Score}{Number}.
\]
\end{document}
