\documentclass[oneside,twocolumn]{article}
\usepackage{mathtools}

%Front matter
\title{Attributes tagging for Degree college upto to UG.}  
\date{Thurday 30, May 2019}
\author{Kipa Tachak}

\setcounter{secnumdepth}{0}

\begin{document}
\maketitle

\section{About the document}
This file documents the process and rationale of designating
particular scoring method for the asset 'Degree college upto
UG'.

\section{Attributes for scoring the asset}
\begin{itemize}
\item General Information
  \begin{itemize}
  \item Departments
  \item Job Placement
  \item Fees
  \end{itemize}
\item Organisations
  \begin{itemize}
  \item Student Union
  \item Societies
  \end{itemize}
\item Partnerships
  \begin{itemize}
  \item Affiliations
  \end{itemize}
\item Scholarships
  \begin{itemize}
  \item Scholarship Oppurtunities
  \end{itemize}
\end{itemize}

\section{Overall asset scoring}
The score for the asset is calculated as,
\begin{align*}
	\text{score} &= w_{GI} \times GI + w_{Org} \times Org \\
		     &+ w_{Pr} \times Pr + w_{Sc} \times Sc.
\end{align*}
where,
\begin{align*}
	GI &= \text{score from general information.} \\
	Org &= \text{score of variable organisations.} \\
	Pr &= \text{score of variable partnerships.} \\
	Sc &= \text{score of variable scholarships.}
\end{align*}

\section{Attributes scoring and the process}
\subsection{General Information}
The general information variable has three sub-paramters, i.e.,
Departments, Job placement, and Fees. The scoring will be
done as follows,
\begin{align*}
	GI &= w_{Dpts} \times Dpts \\
	   &+ w_{Jps} \times Jps \\
	   &+ w_{Fees} \times Fees.
\end{align*}
where,
\begin{align*}
	Dpts &= \text{score of department variable.} \\
	Jps &= \text{score of job placement variable.}
\end{align*}

The scoring of sub-paraters are given below.

\subsubsection{Departments}
The input for this variable is a a string containing the
names of the depatment present in the asset. Assuming, we
can separate the department names, we can classify the
departments into the field they are related to. Then, first
we need to set the weight of a department class.

For the purpose of our analysis, we consider the following
classes of departments and their weights.
\begin{center}
  \begin{tabular}{c|c}
    \hline
    Department type & Weight \\ \hline
    Engineering     & 6 \\
    Medicine        & 6 \\
    Biology         & 5 \\
    Physics         & 5 \\
    Chemistry       & 5 \\
    Social Sciences & 3 \\
    Arts            & 3 \\
    Law             & 4 \\ \hline
  \end{tabular}
\end{center}

Also, a threshold has to be set for the optimum number of
department for each department class. Above, the threshold
the full score for a department class will be added.

Depending on the number of matched classification for each
class from an input, we will calculate the score as follows,

\begin{align*}
  \text{Dept}_{\text{score}} &= \dfrac{w_1 \times \dfrac{D_1}{thresh}}{w_1 \times w_2 \times \dots} \dfrac{+ w_2 \times \dfrac{D_2}{thresh} +}{} \\
  &\dots \dfrac{+ w_n \times \dfrac{D_n}{thresh}}{\dots \times w_n}
\end{align*}

where,
\begin{align*}
  D_1, D_2, \dots, D_n =& \text{ Match count of department classes.} \\
  w_1, w_2, \dots, w_n =& \text{ Weight of each department class.} \\
  \text{thresh} =& \text{ Optimum match count for a} \\
                &\text{department class.}
\end{align*}
Note that, if the match count is above the threshold then
the department class match count \(D_i\) is set to \(thresh\).

\subsubsection{Job Placement}
This is a binary yes/no type of question. The input will be
either yes or no depending on whether the college provides
placement i.e., has a placement department dedicated for the
taskor. The variable describes the presence/absence of
placment and the quality of placement in the training college.
\paragraph{Scoring}
Scoring is done as follows,
\begin{align*}
	S &= 0, \text{if selected no} \\
     	  &= 1, \text{if selected yes}
\end{align*}
where, \(S = \text{denotes the presence/absence of placement}\).

If there some placement facilites then score will be based on
the following criterias:
\begin{enumerate}
  \item The average number of companies that come to the
  campus each year: \\
  \textbf{Suggestion}: This should be numeric type question. \\

  We assume that this will be a numeric type question. A numerical
  values representing the average number of companies will be
  given as input.

  A threshhold value will be set. If the input is equal or greater
  that the threshold then the asset will have maximum score
  for this criteria. 

  The score will increase in a non-linear fashion. i.e. score
  gaps between consecutive input values will differ depending
  on where they lie in the input range. For smaller values, the
  gap will be larger. But for larger values the score gaps will
  be smaller.

  Hence, the scoring equation is given as follows,
  \[
	C = \dfrac{\sqrt{NC}}{\sqrt{threshold}}.
  \]
  where, \(C\) is the score and, \(NC\) is the average number of companies
  visiting.

  \item The percentage of students that get placed: \\
  \textbf{Suggestion}: There should be more choices, so that
  score could be finegrained. \\  

  This is a multiple chocice single select question. There are
  three choices available for ranges in terms of percentage
  students that get placed. The score for each choice is given
  below.
  \begin{center}
    \begin{tabular}{c | c | c} \hline
      Placement percentage & score & Description \\ \hline
      0-25\%		     & 0.20  & Good score \\
      25-50\%		     & 0.70  & Better Score \\
      \(>\)50\%	             & 1.00  & Best Score \\ \hline
    \end{tabular}
  \end{center}
  \end{enumerate}

The overall score for job placement variable is given as follows,
\[
	J = S \times (w_c \times C + w_p \times P).
\]
where,
\begin{align*}
	S &= \text{value for selected.} \\
	C &= \text{score for average companies visiting.} \\
	P &= \text{score for student placement.} \\
	w_c &= \text{weight for companies visiting.} \\
	w_p &= \text{weight for student placement.} \\
	J &= \text{Score for job placement variable.}
\end{align*}
\subsubsection{Fees}
The input to this variable will be numerical, specifying the
average fee per student.
\paragraph{Scoring}
The score for fees will decrease as the fees amount goes high.
The score will be 0 (minimum value) if the fee amount crosses
a certain threshold. Also, score will be 1 (maximum value) if 
the college doesn't require any fees.

The scoring equations are given as below,
\begin{align*}
  F =& 0, \text{If fee amount greater than or } \\
    &\text{equal to threshold} \\
	  =& 1, \text{If fee amount = 0} \\
	  =& \dfrac{1}{\text{fee amount}}
\end{align*}
where, \(F = \text{Score for fees}\).

\section{Organizations}
This variable quantifies the organisational activities
occuring in the institute.

The variable has two sub-parameters, Student Unions and
Societies. The overall scoring of the variable will be done
as,
\begin{align*}
  \text{score}_{\text{Org.}} &= w_{\text{student union}} \times \text{Student Union} \\
  &+ w_{\text{societies}} \times \text{Societies}
\end{align*}

\subsubsection{Student Union}
The sub-parameter 'Student Union' has two questions.
\begin{enumerate}
\item Is there any student union in the institute?: \\
  Only if there is a student union in the institute we
  proceed to the next question. Hence, we have the
  following equation for this question,
  \begin{align*}
    \text{present}_{Union} =& 0, \text{if not present.} \\
    =& 1, \text{if present.}
  \end{align*}
\item How many active student unions are there?: \\
  This would have a numerical answer denoting the number
  of active student unions.
\end{enumerate}

\paragraph{Scoring}
A threshold have to be established at which or above,
the student union sub-parameter will be assigned full
score.

The scoring of student union sub-parameter is given as,
\[
\text{score}_{\text{union}} = \text{present}_{union} \times 
\dfrac{\text{no. of unions}}{\text{threshold}}
\]

\subsubsection{Societies}
This sub-parameter only has one question asking whether
there is any societal activities happening e.g., NSS, NCC, etc.

So the scoring proceeds as follows,
\begin{align*}
  \text{present}_{\text{societies}} =& 0, \text{if not present.} \\
  =& 1, \text{if present.}
\end{align*}

\subsubsection{Job Placement}
This is a binary yes/no type of question. The input will be
either yes or no depending on whether the college provides
placement i.e., has a placement department dedicated for the
taskor. The variable describes the presence/absence of
placment and the quality of placement in the training college.
\paragraph{Scoring}
Scoring is done as follows,
\begin{align*}
	S &= 0, \text{if selected no} \\
     	  &= 1, \text{if selected yes}
\end{align*}
where, \(S = \text{denotes the presence/absence of placement}\).

If there some placement facilites then score will be based on
the following criterias:
\begin{enumerate}
  \item The average number of companies that come to the
  campus each year: \\
  \textbf{Suggestion}: This should be numeric type question. \\

  We assume that this will be a numeric type question. A numerical
  values representing the average number of companies will be
  given as input.

  A threshhold value will be set. If the input is equal or greater
  that the threshold then the asset will have maximum score
  for this criteria. 

  The score will increase in a non-linear fashion. i.e. score
  gaps between consecutive input values will differ depending
  on where they lie in the input range. For smaller values, the
  gap will be larger. But for larger values the score gaps will
  be smaller.

  Hence, the scoring equation is given as follows,
  \[
	C = \dfrac{\sqrt{NC}}{\sqrt{threshold}}.
  \]
  where, \(C\) is the score and, \(NC\) is the average number of companies
  visiting.

  \item The percentage of students that get placed: \\
  \textbf{Suggestion}: There should be more choices, so that
  score could be finegrained. \\  

  This is a multiple chocice single select question. There are
  three choices available for ranges in terms of percentage
  students that get placed. The score for each choice is given
  below.
  \begin{center}
    \begin{tabular}{c | c | c} \hline
      Placement percentage & score & Description \\ \hline
      0-25\%		     & 0.20  & Good score \\
      25-50\%		     & 0.70  & Better Score \\
      \(>\)50\%	             & 1.00  & Best Score \\ \hline
    \end{tabular}
  \end{center}
  \end{enumerate}

The overall score for job placement variable is given as follows,
\[
	J = S \times (w_c \times C + w_p \times P).
\]
where,
\begin{align*}
	S &= \text{value for selected.} \\
	C &= \text{score for average companies visiting.} \\
	P &= \text{score for student placement.} \\
	w_c &= \text{weight for companies visiting.} \\
	w_p &= \text{weight for student placement.} \\
	J &= \text{Score for job placement variable.}
\end{align*}
\subsubsection{Fees}
The input to this variable will be numerical, specifying the
average fee per student.
\paragraph{Scoring}
The score for fees will decrease as the fees amount goes high.
The score will be 0 (minimum value) if the fee amount crosses
a certain threshold. Also, score will be 1 (maximum value) if 
the college doesn't require any fees.

The scoring equations are given as below,
\begin{align*}
  F =& 0, \text{If fee amount greater than or } \\
    &\text{equal to threshold} \\
	  =& 1, \text{If fee amount = 0} \\
	  =& \dfrac{1}{\text{fee amount}}
\end{align*}
where, \(F = \text{Score for fees}\).

\section{Organizations}
This variable quantifies the organisational activities
occuring in the institute.

The variable has two sub-parameters, Student Unions and
Societies. The overall scoring of the variable will be done
as,
\begin{align*}
  \text{score}_{\text{Org.}} &= w_{\text{student union}} \times \text{Student Union} \\
  &+ w_{\text{societies}} \times \text{Societies}
\end{align*}

\subsubsection{Student Union}
The sub-parameter 'Student Union' has two questions.
\begin{enumerate}
\item Is there any student union in the institute?: \\
  Only if there is a student union in the institute we
  proceed to the next question. Hence, we have the
  following equation for this question,
  \begin{align*}
    \text{present}_{Union} =& 0, \text{if not present.} \\
    =& 1, \text{if present.}
  \end{align*}
\item How many active student unions are there?: \\
  This would have a numerical answer denoting the number
  of active student unions.
\end{enumerate}

\paragraph{Scoring}
A threshold have to be established at which or above,
the student union sub-parameter will be assigned full
score.

The scoring of student union sub-parameter is given as,
\[
\text{score}_{\text{union}} = \text{present}_{union} \times 
\dfrac{\text{no. of unions}}{\text{threshold}}
\]

\subsubsection{Societies}
This sub-parameter only has one question asking whether
there is any societal activities happening e.g., NSS, NCC, etc.

So the scoring proceeds as follows,
\begin{align*}
  \text{present}_{\text{societies}} =& 0, \text{if not present.} \\
  =& 1, \text{if present.}
\end{align*}

\section{Partnerships}
The variable has only one sub-parameter 'Affiliations'.
Hence, the scoring is to be done as follows,
\[
\text{score}_{\text{partnerships}} = \text{score }_{\text{Affiliations}}.
\]
\subsection{Affiliations}
The input for this sub-parameter is a a string containing the
affiliations of the insitute. Assuming, we can classify the
affiliations into sets of classes with their own rank. We can
set a threshold for the optimum number of affiliations. If the
number of affiliations for a particular affilation class is
greater than the threshold, then this sub-parameter will get the
full value (i.e. weight) for that affiliation class.

Depending on the number of matched classification for each
class from an input, we will calculate the score as follows,

\begin{align*}
  \text{Affiliation}_{\text{score}} &= \dfrac{w_1 \times \dfrac{A_1}{thresh}}{w_1 \times w_2 \times \dots} \dfrac{+ w_2 \times \dfrac{A_2}{thresh} +}{} \\
  &\dots \dfrac{+ w_n \times \dfrac{A_n}{thresh}}{\dots \times w_n}
\end{align*}

where,
\begin{align*}
  A_1, A_2, \dots, A_n =& \text{ Match count of affilation} \\
  & \text{ classes for an input string.} \\
  w_1, w_2, \dots, w_n =& \text{ Weight of each affiliation class.} \\
  \text{thresh} =& \text{ Optimum match count for a} \\
                &\text{department class.}
\end{align*}
Note that, if the match count is above the threshold then
the affiliation class match count \(A_i\) is set to \(thresh\).

\section{Scholarships}
This variable has only one sub-parameter, 'Scholarship Opportunities'.
The question for scholarship oppurtunities is a binary yes/no
question. Hence, value for scholarship is given as,
\begin{align*}
  \text{score}_{\text{Scholarhips}} =& 0, \text{if no scholarhsip opportunities.} \\
    =& 1, \text{if scholarship oppurtunities present}.
\end{align*}
\end{document}
